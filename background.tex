\section{Background}
First, we overview the basic formalism of \sys and present its relationship to related work.

\subsection{Data Transformations}
We focus on data transformations that concern a single relational table. 
Let $R$ be an instance over a set of attributes $A$, and let $\mathcal{R}$ denote the set of all possible instances over $A$.
A data transformation is a function that maps an instance $R \in \mathcal{R}$ to a new instance with the same schema $R' \in \mathcal{R}$: 
\[T(R): \mathcal{R} \mapsto  \mathcal{R}\]
Data transformations preserve schema and data type, and have a binary operator $\circ$, which denotes composition:
\[
(T_i \circ T_j)(R) =  T_i(T_j(R))
\]
Let $\Sigma$ define a set of distinct transformations $\{T_1, T_2,..,T_N\}$.
Let $\Sigma^*$ be the set of all finite sub-sequences of $\Sigma$, i.e., $T_i(T_j(R))$.
A formal language $L$ over  $\Sigma$ is a subset of $\Sigma^*$.


\vspace{0.5em} \noindent \emph{Example 1: } As an example, given below is a relation with two attributes \textsf{city\_name} and \textsf{city\_code}. 
Assuming unique city names, there should be a one-to-one map between city names and the city code. The problem is that the relation we have is inconsistent.

\begin{table}[ht!]
\centering
\label{my-label}
\begin{tabular}{|l|l|l|}
\hline
\rowcolor[HTML]{000000} 
& {\color[HTML]{FFFFFF} city\_name}            & {\color[HTML]{FFFFFF} city\_code}   \\ \hline
1 & San Francisco                                & SF                                  \\ \hline
2& {\color[HTML]{FE0000} \textbf{New York}}     & NY                                  \\ \hline
3 & New York City                                & {\color[HTML]{FE0000} \textbf{NYC}} \\ \hline
4 & {\color[HTML]{FE0000} \textbf{San Francisc}} & SF                                  \\ \hline
5 & San Jose                                     & SJ                                  \\ \hline
6 & San Mateo                                    & SM                                  \\ \hline
7 & New York City                                & NY                                  \\ \hline
\end{tabular}
\end{table}

\noindent Consider the following function:
\[
\textsf{find\_replace}(\text{source}, \text{target}, \text{attribute})
\]
This function finds all cells with the source string, and replaces it with a target string, in the specified attribute.
If we consider exact string matches and only target strings in the attribute's current domain, there are $61$ possible operations for the example relation.
The set of all find-and-replace operations defines $\Sigma$.
$\Sigma^*$ defines any finite sequences composed of these $61$ operations.
$L$ is any subset of $\Sigma^*$. 
For example, we may only want to consider sequences of length $k$.


\subsection{Search Over Data Transformations}
Every data cleaning problem in \sys is specified by a deterministic finite automaton (DFA). 
A DFA is a 5-tuple:
\[\langle S, A, \delta, s_0\rangle,\]
where $S$ is a set of states that the process can be in, $A$ is a set of inputs that the process can take, $\delta$ is a transition function that takes as input a state and an input and transitions the process to the next state in $S$, and $s_0$ is an initial state of the process.

The set of relations and the language of transformations defines a DFA:
\[\langle \mathcal{R}, \Sigma, \delta, R_{dirty}\rangle, \]
where the states $\mathcal{R}$ is the set of possible instances, the inputs are transformations from $\Sigma$, the transition function updates the instance with the transformation, and the initial state is the dirty instance $R_{dirty}$. Search problems can be defined over the DFA. 
A quality function $Q$ maps an instance $R$ to a scalar where 1 implies it is clean:
\[
Q: \mathcal{R} \mapsto [0,1]
\]
A separable quality function is one that can be expressed as an average over cell-wise quality metrics $q(r,a)$ where 1 implies clean:
\[
Q(R) \propto \sum_{r \in R} \sum_{a \in A} q(r,a)
\]
Given a quality function $Q$, a relation $R$, and a language $L$, find a sequence of transformations $l \in L$ (a sequence of inputs to the DFA) that optimizes the quality function.
\[
\textsf{search}(Q,R,L) = ~ \min_{l \in L} Q( l(R) ).  
\]

\vspace{0.5em} \noindent \textbf{Remark 1: } For readers familiar with stochastic processes, this DFA is equivalent to a deterministic Markov Decision Process (MDP), where the states are $\mathcal{R}$, the actions $\Sigma$, the transition function updates the instance with the transformation, the initial state is the dirty instance $R_{dirty}$, and the reward function is $Q$. MDPs are usually stochastic and as such the optimal solution in general is not a sequence but a function from states to actions.


\vspace{0.5em} \noindent \textbf{Remark 2: } \textsf{search(Q,R,L)} is a hard problem. In fact, we can show that it is APX-Hard--that is, unless P=NP there does not exist a polynomial time approximation scheme.
Let $R$ be a single-attribute relation of Booleans. Let $L$ be the set of all assignments to a single value.
Given a list of $N$ Boolean clauses over all the boolean variables, let $Q$ assign to each record one minus the fraction of clauses that evaluate to true. This formulation is equivalent to MAX-SAT and solution to the optimization problem.


\vspace{0.5em} \noindent \emph{Example 2: } 
One can model this constraint with two functional dependencies: $\textsf{city\_name} \rightarrow \textsf{city\_code}$ and $\textsf{city\_code} \rightarrow \textsf{city\_name}$.
With these functional dependencies, identifying inconsistencies is efficient -- it requires querying for the set cities that map to more than one city code, and vice versa. 
$Q(R)$ is a separable quality metric where each cell is assigned a 0  if it contains a violating key. One solution to the search problem is:
\begin{lstlisting}
find_replace(New York, New York City, city_name)
find_replace(San Francisc, San Francisco, city_name)
find_replace(NYC, NY, city_code)
\end{lstlisting}

\section{Taxonomy of Related Systems}
Since the beginning of data management, several research and commercial systems have been proposed to improve data cleaning efficiency and accuracy (see~\cite{rahm2000data} for a survey).
Next, we describe how different data cleaning paradigms can be cast as a sequential search problem.

\subsection{Constraint-based Systems}
The classical model for data cleaning is using integrity constraints. 
The data scientist describes the relationships between attributes in terms of functional dependencies, conditional functional dependencies, denial constraints, and other types of logical constraint constraints.
Then, if any of these constraints are violated, the system will search over updates to the inconsistent database instance to enforce the constraints.
Recent example systems include NADEEF~\cite{DBLP:conf/sigmod/DallachiesaEEEIOT13}, LLunatic~\cite{geerts2013llunatic}, Holistic Data Cleaning~\cite{chu2013holistic}, and Big Dansing~\cite{khayyat2015bigdansing}.
As in the running example, we can construct a quality function that marks every cell that is in violation with a 0 and every cell that is not with a 1.
The allowed language would be replacing any cell with another value in the attribute domain.

While prior work are roughly speaking equivalent in expressiveness, they each have a different solution algorithm.
For example, Dallachiesa et al. uses a SAT solver to enforce the constraints~\cite{DBLP:conf/sigmod/DallachiesaEEEIOT13}, Chu et al. uses a iterative algorithm that walks along a hypergraph~\cite{chu2013holistic}, and Geerts et al. is based on a fixed-point iteration~\cite{geerts2013llunatic}.
Our goal with \sys is to reduce the complexity of such systems, by consolidating to a single general-purpose search algorithm and an API for specifying optimizations specific to particular problem classes.
This means that the approach is more general and can handle novel combinations of data cleaning operations, e.g., denial constraints and numerical outliers.

\subsection{Model-based Systems}
Another class of systems uses statistical properties to define data errors, which is also called Quantitative Data Cleaning (see survey by Hellerstein~\cite{hellerstein2008quantitative}).
A user can define a statistical model that the data should conform to, e.g., all numerical values must be concentrated with $p$ standard deviations from the mean. 
More generally, one can use a model to score records to see how likely those value combinations are.
When presented with data that explained well by this model, the system tries to replace the value.
Example systems include Eracer~\cite{eracer}, SCARED~\cite{yakout2013don}, and dBOOST~\cite{pit2016outlier}.

Such scoring functions naturally fit into the model of \sys.
A statistical model can define the quality function and the user can chose preferred behavior for abnormal records.
For example, one can chose to delete or impute with a default value. Each cell is scored by a an anomaly detector which marks a percentage confidence that the element is an anomaly (e.g., by the distance to the nearest neighbor). The quality function is the average score over the entire table.
The search problem is to find a sequence of transformations that reduces the anomaly detection scores.

Similar techniques can be applied to non-numerical data if an appropriate featurization is used.
One approach is to borrow recent results from Natural Language Processing using Neural Networks to first embed the records in a vector-space and then apply numerical outlier detection techniques. The \textsf{word2vec} model \cite{mikolov2013distributed} is one such approach.
Using large amounts of unannotated plain text, \textsf{word2vec} learns relationships between words automatically with a Neural Network that predicts the occurrence of nearby words.
 Each word is assigned a vector in the vector space such that words that share common contexts (i.e., occur in the same document) in the corpus are located in close proximity to one another in the space.
 This vector space captures semantic relationships between words.
 
Each record is treated as a document and each attribute is treated as word.
 The model is then trained using all of the records in the training dataset.
 Thus, for each attribute value we have a vector.
 To featurize a record, we concatenate these vectors together.
 Therefore, for each record $r$ there is an associated vector $r_v$.
  Like the NLP application, this vector space captures semantic relationships between records.
  We can define the same anomaly detection objective as above over these features.
  
\subsection{Programming-by-Demonstration}
Finally, \sys is very related to systems that apply programming-by-demonstration (PbD) approach to data preparation problems.
This approach was most notably proposed in the Data Wrangler project, where a human provided initial examples of how to transform a set of semi-structured tuples into a structured schema~\cite{wrangler,trifacta}.
The system performed a search to find a sequence of transformations to best reproduce the humans demonstration.
Recently, this basic approach has been extended in the Foofah system~\cite{jin2017foofah}.

Such approaches are clearly linked to \sys. The quality function is a score on each record on how accurately it matches a manually cleaned gold-standard. \sys can additionally couple automated and PbD approaches. 
Consider the example table as before (where managers can't earn less than employees), but this time  a human provides an example repair.
The quality function measures the degree to which the table matches the human examples after applying the data transformations.








