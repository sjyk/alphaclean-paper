\section{Conclusion and Future Work}
In summary, \sys poses data cleaning as a planning problem over a language of data transformations.
\sys generalizes constraint-based, statistical, and demonstration-based data cleaning.
The prevailing wisdom in the design of data cleaning algorithms is to exploit the details of specific problem rather than considering the most general cases, and our experiments suggest that this a general framework like \sys can achieve parity in terms of accuracy.
While the serial implementation of \sys can be much slower than the competitor specialized frameworks, \sys can be efficiently distributed to significantly reduce the gap.

These results should be considered a proof-of-concept that such a data cleaning \sys can be built around the recent results in AI. 
However, to us, these results are still counter-intuitive, and raise a number of speculative questions for the future: (1) are specialized systems overly engineered for worst-case guarantees and perhaps real-datasets are not that pathalogical, (2) maybe the benchmarks that we consider in data cleaning are too easy too brute-force, (3) what are the failure modes and corner cases of \sys in real data.
We hope to consider these problems in future work, as well as extending the system to novel settings.
In particular, we are interested in \sys as a middleware layer for data visualization.
A user can manipulate data in a visual UI and these manipulations can be translated into a quality function.

