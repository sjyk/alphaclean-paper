\section{Related Work}
Since the beginning of data management, several research and commercial systems have been proposed to improve data cleaning efficiency and accuracy (see~\cite{rahm2000data} for a survey).
Over the past few years, there have been several significant data cleaning advances in scalability~\cite{wang1999sample, khayyat2015bigdansing, altowim2014progressive}.
However, \emph{machine time} is only part of the story and the \emph{human time} of data cleaning is known to be significant.
\sys aims to reduce the human time in writing data cleaning scripts by allowing the data scientist to model the data at a high-level and the system searches for the appropriate low-level transformations.

\vspace{0.25em} \noindent \textbf{Learning in data cleaning: } There are a number of other works that use machine learning to improve the efficiency and/or reliability of data cleaning~\cite{DBLP:journals/pvldb/YakoutENOI11,yakout2013don,gokhale2014corleone}.
For example, Yakout et al. train a model that evaluates the likelihood of a proposed replacement value \cite{yakout2013don}.
Another application of machine learning is value imputation, where a missing value is predicted based on those records without missing values.
Machine learning is also increasingly applied to make automated repairs more reliable with human validation \cite{DBLP:journals/pvldb/YakoutENOI11}.
Human input is often expensive and impractical to apply to entire large datasets.
Machine learning can extrapolate rules from a small set of examples cleaned by a human (or humans) to uncleaned data \cite{gokhale2014corleone, DBLP:journals/pvldb/YakoutENOI11}.
This approach can be coupled with active learning \cite{DBLP:journals/pvldb/MozafariSFJM14} to learn an accurate model with the fewest possible number of examples.
Holoclean~\cite{rekatsinas2017holoclean} leverages machine learning to validate repairs with a probabilistic graphical model.
\sys uses machine learning in the synthesis process to prune search branches.


\vspace{0.25em} \noindent \textbf{Generalized Definitions of Data Quality: } Also relevant to \sys, are works that consider generalized definitions of data quality--beyond integrity constraints.
There is a growing body of literature that studies analysis-driven data cleaning, that is, applying data cleaning in a sufficient way to answer a given query (defining quality flexibly as the accuracy of the query result).
For example, Altwaijry et al.~\cite{altwaijry2015query} describe a technique for resolving a sufficient subset of entities in a database to answer SPJ queries.
Bergman et al. \cite{DBLP:conf/sigmod/BergmanMNT15} proposed identifying errors in selection query results and generating crowd-scoured queries to determine fixes to the base data.
Similarly, work on the consistent query answering problem explored the minimal effort needed to answer a query given a set of integrity constraints over a dirty relation~\cite{DBLP:series/synthesis/2011Bertossi}.
While the work on relational queries is extensive, analytical queries (aggregates, advanced statistical analytics, learning etc.) is less studied.
Projects like ActiveClean~\cite{DBLP:journals/pvldb/KrishnanWWFG16} have studied algorithms for prioritizing user-defined cleaning using the downstream ML model.
\sys is a flexible framework where such cleaning objectives can be modeled easily as different quality functions.

\vspace{0.25em} \noindent \textbf{Languages in Cleaning: } Languages for data transformations are also a well-studied field. The seminal work on this subject was Raman and Hellerstein~\cite{raman2001potter} for schema transformations, and Galhardas et al.~\cite{DBLP:conf/vldb/GalhardasFSSS01}. These ideas were later extended in the Wisteria project~\cite{DBLP:journals/pvldb/HaasKWF015} to parametrize the transformations to allow for learning and crowdsourcing. Algorithmically, \cite{wrangler, jin2017foofah} are the closest relatives to \sys. These systems use human demonstrations of data transformations, and leverage a search over the language to mimic these demonstrations. This process is a special case of \sys.



