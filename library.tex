\section{Library Generation}
The first step of \sys is generating a library $L$ over which we will perform our search.

\textbf{SK. TODO}

\subsection{Transformation API}
Data transformations are specified in terms of templates that can take different types of literal parameters.

\begin{enumerate}
    \item Value
    \item Column
    \item Predicate
    \item Substring
\end{enumerate}

We provide a large set of these templates but users can write their own too.


\subsection{Library Generation}
Given a relation $R$, we populate the library with feasible literal values.

\begin{enumerate}
    \item Library gets large
    \item Hints from quality functions about columns and predicates
    \item Note how these optimizations are similar to those in Scorpion.
\end{enumerate}

\subsection{Blocking}
Another approach to reduce the library size is to partition the relation into smaller logical blocks.

\begin{enumerate}
    \item Can sometimes be inferred from quality function.
    \item Similar to what is done in entity resolution.
\end{enumerate}

