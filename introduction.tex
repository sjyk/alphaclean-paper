\section{Introduction}\label{intro}\sloppy
It is widely known that data cleaning is one of the most time-consuming steps of the data analysis process~\cite{nytimes}.
Designing algorithms and systems to automate or partially automate data cleaning is very challenging and continues to be an active area of research~\cite{DBLP:conf/sigmod/ChuIKW16}.
The primary technical challenge is that real-world data is highly variable, where a given dataset can have multiple different types of errors (e.g., numerical outliers, duplicates, and missing values).
Each type of error is often handled by a different class of algorithms, if not, a completely different system.

Consequently, data cleaning pipelines often span across multiple sub-systems to best address different types of errors~\cite{krishnan2016hilda} making them difficult to tune, debug, and generalize when the data changes~\cite{sculley2014machine,krishnan2016hilda}.
Ideally, designing such pipelines by hand should be avoided and data cleaning should be \emph{declarative}, wherein the analyst specifies a high-level data model (e.g., constraints the data should satisfy) and a system optimally enforces the model.
In other words, the data scientist need not code low-level data transformations but instead focus her effort on specifying a high-level data model.
The idea of declarative data cleaning is not new~\cite{rahm2000data}, but such approaches have classically been restricted to data models specified in subsets of first-order logic and only recently have considered extensions better handle uncertain numerical data~\cite{prokoshyna2015combining}.
The prevailing wisdom is that there is an inherent tradeoff between the expressiveness of the data model and the efficiency of the (approximate) solution algorithm---limited models allow the system designer to exploit specific algorithmic structures.

However, two recent trends, namely, the success of model-free learning in AI and the availability commodity cloud computing encourage us to reconsider this algorithmic philosophy.
Recent progress in planning problems such as AlphaGo~\cite{silver2016mastering} and in Atari video games~\cite{mnih2015human} have shown that a prudent combination Machine Learning and distributed search can approximately optimize very complex black-box objective functions.
The intriguing aspect of these results is that the same algorithm that learns to play an Atari game~\cite{mnih2015human} can be used to train a robot to perform as task~\cite{gu2017deep}.
The underlying optimization algorithm encodes few specifics about the objective or structure of the domain.
Domain knowledge is introduced after the fact in terms of how the problem is featurized or how the planning problem is defined.

We explore the extent to which the same approach applies to data cleaning.
Instead of developing specialized algorithms for each new variety of data model, we should consider a general discrete optimization framework that models almost all data cleaning problems: given a black-box objective function that defines cleanliness, find a sequence of data transformations that maximizes this objective.
The logic for the special cases should introduced as pruning rules (e.g., disallowed sequences) or modifications to the data representation (e.g., clustering certain records together) to solve specific problem instances more efficiently.
And, instead of specially tuned search heuristics, the heuristics are learned incrementally while cleaning the data.
The benefits of this approach are: (1) there are few restrictions on what the user can express, where a more complex objective function simply requires a longer optimization time, (2) the output is a sequence of transformations rather than a clean database instance so the logic of the algorithm is more interpretable and can apply to future data, and (3) it greatly simplifies the solution algorithm since it is not coupled to any particular problem instance.

We present \sys, a new system for data cleaning.
\sys is given an objective function that models the data quality of a relation.
We provide an API that translates existing error models such as Functional Dependencies, Denial Constraints, Statistical Models into an objective function.
\sys is also given a parametrized set of data cleaning operators that transform relations.
Given these two pieces, the objective and the transformation language, \sys incrementally builds a data cleaning program to optimize objective function.
\sys is initialized with the identity operation and repeatedly expands the most promising sequences seen so far (i.e., composing it with a new transformation from the language).
The expansion policy, or the choice of which sequence to expand and how, is where we can exploit the structure of real-world datasets.
For example, if we clean data in small partitioned blocks (common in many data cleaning systems), we can incrementally build a classifier that can score expansions that are likely to be successful--an approach analogous to that used in AlphaGo~\cite{silver2016mastering}.
Similarly, if we know that the objective was generated from a functional dependency, we can restrict the expansions to only those transformations that modify cells specifically in violation. 

The goal of our experiments is to show that we can achieve parity with state-of-the-art constraint, statistical, and quantitative data cleaning systems in terms of accuracy for a reasonable runtime.
On two datasets considered in prior data cleaning work of Flight arrival times and a Physician registry, \sys achieves a similar precision and recall to a recently proposed Denial Constraint system called HoloClean~\cite{rekatsinas2017holoclean}. 
Similarly, we applied \sys to numerical outlier problems and compared to the Minimum Covariance Determinant (MCD) algorithm, which is the basis of another recent system called Macrobase~\cite{bailis2016macrobase}.
Outliers removed by \sys  improved the accuracy of a downstream predictive models more significantly than MCD (5\% prediction accuracy improvement on the US Census Dataset) and a (4\% on an EEG dataset). 
But more importantly, this is achieved with a core codebase of 2000 lines of python code, and readily amenable to aggressive systems optimizations such as parallelization, caching, and incremental maintenance.
In all of our experiments, we present the degree of parallelization need to match the the runtime of a state-of-the-art competitor.

In short, \sys enables a cleaning system that directly leverages the user's domain expertise to articulate high level cleaning goals and automatically chooses a sequence of transformations.  Although it is still possible to design pathological settings for our approach, our experiments show strong evidence that \sys is comparable or better in terms of accuracy than specialized cleaning systems, can meet competitive run times to those systems.
It additionally easily extends to datasets with a mixture of error types---such as formatting errors and integrity constraint violations.   In addition, we believe the high level interface can accelerate end-to-end data cleaning for modern data science applications, while leveraging the substantial parallelization opportunities brought by the cloud. 





\if{0}
Data cleaning usually involves tedious manual specification of transformation rules.
For example, a data scientist might write a rule that maps every record with a \textsf{country} attribute ``United States'' to ``United States of America''.
These collections of rules quickly grow in size, and if they are developed in an ad-hoc way, they can be brittle and hard to maintain~\cite{krishnan2016hilda}.
Overly specific rules may not apply to future data, and overly general rules, might introduce unwanted side-effects.
Designing accurate transformation rules is a painstaking process, which is widely reported to be one of the most effort-intensive steps in data science~\cite{nytimes}.

We explore to what extent such transformation rules can be learned through a process of automatic trial-and-error on a dataset.
The system simulates different sequences of data transformations and scores the result according to a user-specified data quality objective function.
On its own, this is an inefficient way to search to search the set of transformation rules.
However, we can leverage Machine Learning to learn a strong pruning heuristic incrementally.
Transformations are often parametrized by literal values from the database (e.g., find string X and replace with string Y).
If we clean data in small partitioned blocks, we can incrementally build a classifier that infers common patterns in these literals, such as whether the strings tend to be similar or tend to be different or the attributes that are likely to be touched.
For example, consider the problem above where a data scientist is resolving inconsistencies in a \textsf{country} attribute to satisfy an integrity constraint.
As we iterate through the distinct \textsf{country} values and simulate possible replacement values, it might become clear that the source string have to be close to the target strings in string similarity, i.e., ``United States'' is more likely to be replaced by ``United States of America'' than ``Zimbabwe''.
Learning such a relationship automatically avoids hand coded heuristics which have to consider the complex relationship between how the analyst describes data quality and how exactly the transformations are defined.

This basic algorithm, optimizing a sequence of black-box transformations over a black-box data quality function, has a number of important advantages: (1) it is highly expressive as it can model a wide range of data cleaning formalisms from integrity constraints to quantitative data cleaning in the same framework, (2) it is relatively easy to parallelize the search, and (3) the result of the algorithm is not only a cleaned database instance but transformations that can be applied to future data.
This formalism casts data cleaning as a planning problem; analogous to the algorithms used AI, Robotic Planning, and Control.
Similar to the way one plans out a sequence of chess moves in AI to gain a strategic board position, we can think of data cleaning as planning out a sequence of data transformations to maximize the score on a data quality function.
And as in chess, where one cannot perfectly anticipate the opposing player's moves, in data cleaning we may not have a strong \emph{a priori} model of how a user-defined transformation rule modifies the data.
As a result, the algorithm and search heuristics cannot make strong assumptions about the anticipated structure of any search instance.
This sort of an approach is increasingly attractive because recent results in AI demonstrate scaling these search problems to  high branching-factor domains even when few assumptions are made.
Recent algorithms have been shown to match or exceed human performance in domains such as Go~\cite{silver2016mastering} and in Atari video games~\cite{mnih2015human}.
As in many classical data cleaning problems, an optimal solution to AI search problems is very hard to discover, but pragmatically leverage distributed computing and pruning rules learned from data can tractably find acceptable solutions.

We use this algorithm as a starting point for a new data cleaning system called \sys.
We designed an API that takes classical data cleaning problem specifications, such as integrity constraints, gold-standard manually cleaned data, and statistical models, and translates those specifications into an iterative learning search problem. 
Of course, we often need special-case optimizations that prune unproductive search branches to make the runtime more competitive with special case systems.
We provide a model where pruning rules are specified as regular expressions over a formal language of transformations and can be static (i.e., fixed before execution) and dynamic (i.e., inferred from properties of the dirty database instance).
The search algorithm we use is a memory-bounded best-first search which maintains the subset of the search frontier that can fit in memory. 
This algorithm can be parallelized, distributed, and can cache repeated computations.
\fi



