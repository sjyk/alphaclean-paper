\section{Introduction}\label{intro}\sloppy



It is widely known that data cleaning is one of the most time-consuming steps of the data analysis process~\cite{nytimes}, and
designing algorithms and systems to automate or partially automate data cleaning continues to be a rich and active area of research~\cite{DBLP:conf/sigmod/ChuIKW16}.
What makes data cleaning technically challenging is that real-world data is highly variable: a given dataset can manifest a multitude of different types of errors (e.g., numerical outliers, constraint violations, duplicates, etc), and the types of errors that are relevant to the user depends on the intended down-stream application that uses the cleaned dataset.
Beyond these technical challenges, data cleaning is fundamentally a highly iterative expert-intensive process that relies on the domain expert to define {\it what} it means to be a clean dataset and {\it how} the data is allowed to be cleaned.
For these reasons, it is important that data cleaning systems are designed to make best use of the domain expert's scarce resources.

% Each type of error is often handled by a different class of algorithms, if not, a completely different system.
% This results in data cleaning systems that are optimized for a narrow set of problems and are not expressive enough to cover all of the errors in manifest in the data.

Unfortunately, a recent survey suggests that industrial data cleaning approaches are far from this user-centric goal, due to a combination of specialized solutions, system complexity, and solution idiosyncrasies.  First, each type of error is often handled by a different class of algorithms, if not, a completely different system.  This leads to implementation-specific optimizations that are non-trivial to transfer to new classes of data errors.   Second, pipelines are a patchwork of custom scripts and multiple specialized systems~\cite{krishnan2016hilda} taht require considerable technical expertise to understand and stitch together.  Third, many data cleaning systems directly modify the input dataset, whereas in practice, users want more than a clean dataset---they want to understand how the data was cleaned, have control over the types of data cleaning operations that are used, and ultimately clean {\it new} data.  In short, these limitations make it difficult for non-technical experts to take control of the cleaning process, and for technical experts to rapidly create, tune, debug, and generalize their data cleaning solution to new problem domains~\cite{sculley2014machine,krishnan2016hilda}.

In this paper, we propose to reframe data cleaning as a planning and optimization problem.  In planning, the system is given a dsecription of a state space and actions that transition between states, and generates a sequence of actions (a plan) that maximizes a quality function. The classic application is a computer chess program that must plan a sequence of chess moves that change the state of the board in order to maximize the likelihood of a checkmate. Although simple small problems were solvable by exhaustively searching through a manually pruned search space, the general formulation of the problem leads to a tremendously large search space.  For decades, it was unclear whether planning could scale to generate high quality plans for more complex problems such as Go or self-driving cars.

However, we are optimistic about a planning approach to data cleaning based on th erecent success in Go playing (AlphaGo) and self driving cars.  In both applications, their success have been drived by three core innovations.  First, planning algorithms have improved and converged on a varitaino of best-first esarch that works well across many applications.  Second, modern cloud and multicore processing have made massive parallelization economically feasible so that modern planning algorithms can simply search orders of magnitude more states.  Third, and most importantely, advances in machine learning have made it possible to automatically learn pruning functions by identifying and using signals in candidate plans to estimate its expected quality.  This has effectively transformed hand crafting pruning rules with feature engineering.  For instance, in AlphaGo, the pruning heuristics were learned using a neural network via self-play~\cite{}, however even simple linear prediction models work well by identifying and exploiting the structure of the planning problem.

To this end, we propose \sys.  
We explore the extent to which the same approach applies to data cleaning.
Instead of developing specialized algorithms for each new variety of data model, we should consider a general discrete optimization framework that models almost all data cleaning problems: given a black-box objective function that defines cleanliness, find a sequence of data transformations that maximizes this objective.
The additional logic for the special cases should layered onto the general framework as pruning rules (e.g., disallowed sequences) or modifications to the data representation (e.g., clustering certain records together) to solve specific problem instances more efficiently.
One unique contribution of this paper, inspired by AlphaGo, is that search heuristics are learned from the data to avoid users having to carefully tune the algorithm for each use-case.
As more data is cleaned, the system learns common transformation patterns that can be reused in the future. 
The benefits of this approach are: (1) there are few restrictions on what the user can express, (2) the output is a sequence of transformations rather than a clean database instance so the logic of the algorithm is more interpretable and can apply to future data, and (3) it greatly simplifies the solution algorithm since it is not coupled to any particular problem instance.

We present \sys, a new system for data cleaning.
\sys is given an objective function that models the data quality of a relation.
We provide an API that translates existing data cleaning formalisms such as Functional Dependencies, Denial Constraints, Statistical Models into a compatible objective function.
\sys is also given a parametrized set of data cleaning operators that transform relations.
Given these two pieces, the objective and the transformation language, \sys incrementally builds a data cleaning program to optimize objective function.
\sys is initialized with the identity operation and repeatedly expands the most promising sequences seen so far (i.e., composing it with a new transformation from the language).
The expansion policy, or the choice of which sequence to expand and how, is where we can exploit the structure of real-world datasets.
For example, if we clean data in small partitioned blocks (common in many data cleaning systems), we can incrementally build a classifier that can score expansions that are likely to be successful--an approach analogous to that used in AlphaGo~\cite{silver2016mastering}.
Similarly, if we know that the objective was generated from a functional dependency, we can restrict the expansions to only those transformations that modify cells specifically in violation. 

The goal of our experiments is to show that such an architecture can achieve parity with state-of-the-art constraint, statistical, and quantitative data cleaning systems in terms of accuracy--despite the generality.
On two datasets considered in prior data cleaning work of Flight arrival times and a Physician registry, \sys achieves a similar precision and recall to a recently proposed Denial Constraint system called HoloClean~\cite{rekatsinas2017holoclean}. 
Similarly, we applied \sys to numerical outlier problems and compared to the Minimum Covariance Determinant (MCD) algorithm, which is the basis of another recent system called Macrobase~\cite{bailis2016macrobase}.
Outliers removed by \sys  improved the accuracy of a downstream predictive models more significantly than MCD (5\% prediction accuracy improvement on the US Census Dataset) and a (4\% on an EEG dataset). 
But more importantly, this is achieved with a core codebase of 2000 lines of python code, and readily amenable to aggressive systems optimizations such as parallelization, caching, and incremental maintenance.
In all of our experiments, we present the degree of parallelization need to match the the runtime of a state-of-the-art competitor.

In short, \sys enables a cleaning system that allows the user's to articulate high level cleaning goals and automatically chooses a sequence of transformations.  
It is important to note that there are clear limitations in this design.
In some sense, \sys trades off generality for runtime, and it is clearly slower than specialized data cleaning without significantly more resources.
Second, \sys sacrifices many of the provable properties achieved in specialized systems (e.g., constraint satisfaction). 
However, our experiments suggest that in practice \sys is comparable or better in terms of accuracy than specialized cleaning systems, can meet competitive run times to those systems with sufficient parallelization. In addition, we believe the high level interface significantly reduces the human burden in data cleaning for modern data science applications, enabled by the substantial parallelization opportunities brought by the cloud. 





\if{0}
Data cleaning usually involves tedious manual specification of transformation rules.
For example, a data scientist might write a rule that maps every record with a \textsf{country} attribute ``United States'' to ``United States of America''.
These collections of rules quickly grow in size, and if they are developed in an ad-hoc way, they can be brittle and hard to maintain~\cite{krishnan2016hilda}.
Overly specific rules may not apply to future data, and overly general rules, might introduce unwanted side-effects.
Designing accurate transformation rules is a painstaking process, which is widely reported to be one of the most effort-intensive steps in data science~\cite{nytimes}.

We explore to what extent such transformation rules can be learned through a process of automatic trial-and-error on a dataset.
The system simulates different sequences of data transformations and scores the result according to a user-specified data quality objective function.
On its own, this is an inefficient way to search to search the set of transformation rules.
However, we can leverage Machine Learning to learn a strong pruning heuristic incrementally.
Transformations are often parametrized by literal values from the database (e.g., find string X and replace with string Y).
If we clean data in small partitioned blocks, we can incrementally build a classifier that infers common patterns in these literals, such as whether the strings tend to be similar or tend to be different or the attributes that are likely to be touched.
For example, consider the problem above where a data scientist is resolving inconsistencies in a \textsf{country} attribute to satisfy an integrity constraint.
As we iterate through the distinct \textsf{country} values and simulate possible replacement values, it might become clear that the source string have to be close to the target strings in string similarity, i.e., ``United States'' is more likely to be replaced by ``United States of America'' than ``Zimbabwe''.
Learning such a relationship automatically avoids hand coded heuristics which have to consider the complex relationship between how the analyst describes data quality and how exactly the transformations are defined.

This basic algorithm, optimizing a sequence of black-box transformations over a black-box data quality function, has a number of important advantages: (1) it is highly expressive as it can model a wide range of data cleaning formalisms from integrity constraints to quantitative data cleaning in the same framework, (2) it is relatively easy to parallelize the search, and (3) the result of the algorithm is not only a cleaned database instance but transformations that can be applied to future data.
This formalism casts data cleaning as a planning problem; analogous to the algorithms used AI, Robotic Planning, and Control.
Similar to the way one plans out a sequence of chess moves in AI to gain a strategic board position, we can think of data cleaning as planning out a sequence of data transformations to maximize the score on a data quality function.
And as in chess, where one cannot perfectly anticipate the opposing player's moves, in data cleaning we may not have a strong \emph{a priori} model of how a user-defined transformation rule modifies the data.
As a result, the algorithm and search heuristics cannot make strong assumptions about the anticipated structure of any search instance.
This sort of an approach is increasingly attractive because recent results in AI demonstrate scaling these search problems to  high branching-factor domains even when few assumptions are made.
Recent algorithms have been shown to match or exceed human performance in domains such as Go~\cite{silver2016mastering} and in Atari video games~\cite{mnih2015human}.
As in many classical data cleaning problems, an optimal solution to AI search problems is very hard to discover, but pragmatically leverage distributed computing and pruning rules learned from data can tractably find acceptable solutions.

We use this algorithm as a starting point for a new data cleaning system called \sys.
We designed an API that takes classical data cleaning problem specifications, such as integrity constraints, gold-standard manually cleaned data, and statistical models, and translates those specifications into an iterative learning search problem. 
Of course, we often need special-case optimizations that prune unproductive search branches to make the runtime more competitive with special case systems.
We provide a model where pruning rules are specified as regular expressions over a formal language of transformations and can be static (i.e., fixed before execution) and dynamic (i.e., inferred from properties of the dirty database instance).
The search algorithm we use is a memory-bounded best-first search which maintains the subset of the search frontier that can fit in memory. 
This algorithm can be parallelized, distributed, and can cache repeated computations.
\fi



\if{0}


view data cleaning from a planning~\cite{} and optimization perspective, and show evidence that a simple, general approach to this problem can provide the user with considerable flexibility in terms of cleaning data with different classes of errors, controlling the allowable operations, and customizing the system to new types of data errors.



Ideally, the majority of the data cleaning process should reside in a single system with a  \emph{declarative} interface, wherein the analyst specifies a high-level data model (e.g., constraints the data should satisfy) and a system automatically generates the pipeline to enforce the model.
The idea of declarative data cleaning is not new~\cite{rahm2000data}, but such approaches have classically been restricted to data models specified in subsets of first-order logic and only recently have considered extensions better handle uncertain numerical data~\cite{prokoshyna2015combining}.
The prevailing wisdom is that there is an inherent tradeoff between the expressiveness of the data model and the efficiency of the (approximate) solution algorithm---limited models allow the system designer to exploit specific algorithmic structures.


However, two recent trends, namely, the success of model-free learning in AI and the availability commodity cloud computing encourage us to reconsider this algorithmic philosophy.
Recent progress in planning problems such as AlphaGo~\cite{silver2016mastering} and automatically playing Atari video games~\cite{mnih2015human} have shown that a prudent combination of Machine Learning and distributed search can approximately optimize very complex black-box objective functions.
The intriguing aspect of these results is that the same algorithm, called Deep Q Reinforcement Learning, that learns to play an Atari game~\cite{mnih2015human} can be used on a very different problem, such as training a robot~\cite{gu2017deep}.
The underlying optimization algorithm encodes few specifics about the objective or structure of the domain  and pragmatically finds a reasonable local optimum.
The optimization algorithm is general enough to support a very wide class of problems--at the cost of additional computational resources since it is unaware of instance-specific structure.

\fi
