\section{Introduction}\label{intro}\sloppy
Data cleaning is widely recognized as a major challenge in almost all forms of data analytics~\cite{nytimes}, and analysts report spending upwards of 80\% of their time in cleaning and preparing data before analysis. 
Improperly handled errors can affect the performance and accuracy of downstream applications such as reports, visualizations, and machine learning models.
In response, the research community has developed increasingly sophisticated methods for detecting and automatically repairing errors in large datasets~\cite{dc, rekatsinas2017holoclean, DBLP:journals/pvldb/KrishnanWWFG16, DBLP:conf/sigmod/ChuIKW16, mudgal2018deep, doan2018toward}.
The burden on the analyst is gradually shifting away from the design of hand-written data cleaning scripts to building and tuning complex pipelines of automated data cleaning libraries~\cite{krishnan2016hilda}.

The natural first approach is to apply recent hyperparameter tuning systems for machine learning pipelines and neural network model search~\cite{li2017hyperband, sparks2017keystoneml, baylor2017tfx, golovin2017google, liaw2018tune}, since the choice of cleaning operators and their configuration parameters can be viewed as hyperparameters.
However, these systems are fundamentally ill-suited for data cleaning applications since they model their tuning problems as black-boxes that only make decisions with only query access to the final objective value.
Data cleaning pipelines are typically not this opaque.
Many popular data cleaning systems restrict their cleaning operations to cell replacements~\cite{rekatsinas2017holoclean,DBLP:conf/sigmod/ChuIKW16, DBLP:journals/pvldb/KrishnanWWFG16}.
Since these systems preserve schema (same input and output types), they can be reordered, queried/optimized independently, and ensembled in ways that general machine learning pipelines cannot.
Second, the common objective functions considered in data cleaning, like integrity constraints and goodness-of-fit, are often amenable to incremental computation (can be updated from deltas)~\cite{fan2014incremental}.
These observations suggest rethinking parameter tuning for data cleaning pipelines, and this paper studies the problem with a new framework called \sys.

The main architectural insight is relatively straight-forward: rather than treating the cleaning pipeline as a parameterized black-box, we can assess the fine-grained impact of individual repairs that a data cleaning system will make for each possible parameter setting.
To use \sys, users define data quality functions in terms of weighted sum over SQL aggregates over the input table.  This is a highly expressive interface that can express many popular quality measures including integrity constraint violations~\cite{ilyas2015trends} and numerical outliers~\cite{bailis2017macrobase}, as well as application-specific quality notions such as machine learning training accuracy.
Then, users define a library of desired data cleaning operators and their parameter spaces.
As \sys iterates through the parameter settings, each operator suggests a \emph{candidate} repair to the table.
Separate threads search through the pool of candidates to decide a sequence of repairs to construct (a cleaning pipeline) that improves the quality function.

The search algorithm is implemented as a greedy tree-search that sequences the repairs~\cite{russell2016artificial}.
The space of possible repair sequences is enormous (our experiments encounter branching factors in the millions).
Thus, it is important to avoid fully evaluating a path's quality and expanding unpromising paths.    
\sys dynamically learns a model to avoid executing the pipeline and quality function in order to evaluate a given path, and can be tuned to have a low false positive rate when pruning candidate paths.  
Furthermore, the tree search can be easily parallelized across both candidate paths, as well as across partitions of the dataset based on properties of the quality function.
We use periodic synchronization to update the prediction model across parallel searches and merge transformations that repair disjoint sets of tuples. 
This algorithm is clearly heuristic and \textbf{\emph{our primary hypothesis is that many data cleaning problems rarely encounter pathological cases, and thus a general search algorithm combined with recent search optimization heuristics from AI can efficiently search the space}}.

Recent work has highlighted the need for ensembles of cleaning systems to address real-world data errors~\cite{DBLP:journals/pvldb/AbedjanCDFIOPST16}.  In addition to supporting quality functions that can express mixtures of different types of data errors, \sys exhibits many desirable properties for these use cases.  
First, it can optimize over disparate data cleaning systems (that each address different errors) that are integrated as \sys cleaning operators.  Second, it can identify subtle differences between similar cleaning systems to select the repairs best suited to the quality function.  Even in cases where an existing cleaning system is specifically designed for the errors in the dataset (e.g., integrity constraints), \sys can combine other cleaning operators to further improve the repairs.  Our experiments show that one of the most powerful benefits of \sys comes from the ensembling effects and its natural robustness to redundancy and/or distracting pipeline components.

\smallskip\noindent To summarize our contributions:

\begin{itemize}[leftmargin=*, topsep=0mm, itemsep=0mm]
  \item Implementation of \sys, an extensible data cleaning system that supports highly expressive data quality functions, and automatically generates cleaning pipelines to most improve the quality function.  
  \item A progressive search algorithm that quickly generates acceptable cleaning pipelines, and then improves the pipelines over time.  It generates repairs with $\le 9\times$ higher quality than state-of-the-art hyperparameter tuning frameworks on three benchmark datasets.
  \item Optimizations such as incremental computation of quality functions, parallelization, asynchronous search architecture, and adaptive search space pruning, that reduce runtimes by $20\times$.   
  \item A systematic study of the benefits and limitations of \sys showing robustness to straggler and redundant data cleaning operations.
\end{itemize}



