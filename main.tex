\documentclass{sig-alternate}


\usepackage{enumitem}
\usepackage{framed}
%\usepackage[11pt]{moresize}
\usepackage{cprotect}
\usepackage{enumitem}
\usepackage{listings}
\usepackage{amstext}
\usepackage{amstext}
\usepackage{pdfpages}
\usepackage{alltt}
\usepackage{epstopdf}
\usepackage{xspace,colortbl}
\usepackage[USenglish]{babel}
\usepackage{multirow}
\usepackage[hyphens]{url}
\usepackage{subfigure}
\usepackage{graphicx}%%
\usepackage{amssymb}
\usepackage{fmtcount}
\usepackage{amsfonts}
\usepackage{xspace}
\usepackage{amsmath}
\usepackage{multirow}
\usepackage[mathscr]{eucal}
%\usepackage{psfrag}
\usepackage{colortbl}

\usepackage{amsmath,amssymb}
\usepackage[linesnumbered, ruled,vlined]{algorithm2e}

\usepackage{caption}
\usepackage{graphicx}

\usepackage{bm}
\usepackage[nospace]{cite}
\usepackage{csquotes}
\usepackage{enumitem}
\usepackage{times}

\usepackage{courier}

\lstset{basicstyle=\scriptsize\ttfamily,breaklines=true}
\lstset{framextopmargin=50pt}

\usepackage{cleveref}

\usepackage{balance}

%\linespread{0.99}

\makeatletter
\def\@copyrightspace{\relax}
\makeatother


\DeclareMathOperator*{\argmin}{arg\,min}
\DeclareMathOperator*{\argmax}{arg\,max}
\newcommand*{\QEDB}{\ensuremath{\square}}%



\begin{document}

\setlength{\belowdisplayskip}{3pt} \setlength{\belowdisplayshortskip}{3pt}
\setlength{\abovedisplayskip}{3pt} \setlength{\abovedisplayshortskip}{3pt}
\setlength{\belowcaptionskip}{-10pt}
\selectfont

\newtheorem{theorem}{Theorem}
\newtheorem{example}{Example}
\newtheorem{definition}{Definition}
\newtheorem{problem}{Problem}
\newtheorem{property}{Property}
\newtheorem{proposition}{Proposition}
\newtheorem{lemma}{Lemma}
\newtheorem{corollary}{Corollary}

\newcommand{\detectlib}{\texttt{IsoDetect}\xspace}
\newcommand{\company}{\texttt{Company X}\xspace}
\newcommand{\cond}{\textrm{pred}\xspace}
\newcommand{\dataset}{data set\xspace}
\newcommand{\datasets}{data sets\xspace}
\newcommand{\spview}{\textsf{SPView}\xspace}
\newcommand{\fjview}{\textsf{FJView}\xspace}
\newcommand{\aggview}{\textsf{AggView}\xspace}
\newcommand{\hashfunc}[1]{\textsf{hash}(#1)\xspace}
\newcommand{\hashop}{\textsf{hash}\xspace}
\newcommand{\nsc}{\textsf{NormalizedSC}\xspace}
\newcommand{\rsc}{\textsf{RawSC}\xspace}

\newcommand{\avgfunc}{\ensuremath{\texttt{avg} }\xspace}
\newcommand{\maxfunc}{\ensuremath{\texttt{max} }\xspace}
\newcommand{\minfunc}{\ensuremath{\texttt{min} }\xspace}
\newcommand{\histfunc}{\ensuremath{\texttt{histogram\_numeric} }\xspace}
\newcommand{\countfunc}{\ensuremath{\texttt{count}}\xspace}
\newcommand{\sumfunc}{\ensuremath{\texttt{sum} }\xspace}
\newcommand{\varfunc}{\ensuremath{\texttt{var} }\xspace}
\newcommand{\stdfunc}{\ensuremath{\texttt{std} }\xspace}
\newcommand{\covfunc}{\ensuremath{\texttt{cov} }\xspace}
\newcommand{\corrfunc}{\ensuremath{\texttt{corr} }\xspace}
\newcommand{\medfunc}{\ensuremath{\texttt{median} }\xspace}
\newcommand{\percfunc}{\ensuremath{\texttt{percentile} }\xspace}
\newcommand{\havingfunc}{\ensuremath{\texttt{HAVING} }\xspace}
\newcommand{\selectfunc}{\ensuremath{\texttt{select} }\xspace}
\newcommand{\ratio}{\ensuremath{\rho }\xspace}


\newcommand{\insertion}{\ensuremath{\texttt{INSERT} }\xspace}
\newcommand{\update}{\ensuremath{\texttt{UPDATE} }\xspace}
\newcommand{\delete}{\ensuremath{\texttt{DELETE} }\xspace}

\newcommand{\sysfull}{AlphaClean\xspace}
\newcommand{\sys}{AlphaClean\xspace}
\newcommand{\sysnospace}{AlphaClean}


\newcommand{\tbl}[1]{\textsf{#1}\xspace}
\newcommand{\field}[1]{\textsf{#1}\xspace}
\newcommand{\cost}{\textrm{cost}\xspace}
\newcommand{\ans}{\textsf{ans}\xspace}
\newcommand{\dans}{\Delta\textsf{ans}\xspace}
\newcommand{\cqp}{correction query processing\xspace}
\newcommand{\Cqp}{Correction query processing\xspace}

\newcommand{\reminder}[1]{{{\textcolor{magenta}{\{\{\bf #1\}\}}}\xspace}}
\newcommand{\ewu}[1]{{{\textcolor{blue}{\{\{\bf ewu:\} #1\}}}\xspace}}
\newcommand{\mps}[1]{{{\textcolor{red}{\{\{\bf meelap:\} #1\}}}\xspace}}
\newcommand{\stitle}[1]{\smallskip\noindent\textbf{#1: }}
\newcommand{\ititle}[1]{\smallskip\noindent\textit{#1: }}
\newcommand{\btitle}[1]{\smallskip\noindent\textbf{#1}}


\definecolor{light-gray}{gray}{0.95}
\definecolor{mid-gray}{gray}{0.85}
\definecolor{green}{RGB}{0,176,80}
\definecolor{darkred}{rgb}{0.7,0.25,0.25}
\definecolor{darkgreen}{rgb}{0.15,0.55,0.15}
\definecolor{darkblue}{rgb}{0.1,0.1,0.5}
\definecolor{orange}{RGB}{237,125,49}
\definecolor{blue}{RGB}{68,114,196}
\definecolor{pop}{RGB}{0,21,245}

\newcommand{\white}[1]{{\textcolor{white}{#1}\xspace}}
\newcommand{\blue}[1]{{\textcolor{blue}{{\bf #1}}\xspace}}
\newcommand{\orange}[1]{{\textcolor{orange}{{\bf #1}}\xspace}}
\newcommand{\pop}[1]{{\textcolor{pop}{{\textit{\textbf{#1}}}}\xspace}}
\newcommand{\red}[1]{\textcolor{red}{#1}}
\newcommand{\green}[1]{\textcolor{green}{#1}}
\newcommand{\gray}[1]{\textcolor{light-gray}{#1}}




\newcommand{\specialcell}[2][c]{%
  \begin{tabular}[#1]{@{}c@{}}#2\end{tabular}}

\def\ojoin{\setbox0=\hbox{$\bowtie$}%
  \rule[-.02ex]{.25em}{.4pt}\llap{\rule[\ht0]{.25em}{.4pt}}}
\def\leftouterjoin{\mathbin{\ojoin\mkern-5.8mu\bowtie}}
\def\rightouterjoin{\mathbin{\bowtie\mkern-5.8mu\ojoin}}
\def\fullouterjoin{\mathbin{\ojoin\mkern-5.8mu\bowtie\mkern-5.8mu\ojoin}}

%\setlength{\belowcaptionskip}{-10pt}

%\newcommand{\reminder}[1] {}
\pagestyle{plain}

%\input{coverletter.tex}

%\title{\sys: Declarative Data Cleaning with \\ Learning and Tree-Search}
% \title{\sys: Automatic Data Cleaning For Humans}
% \title{\sys: Automatic Data Cleaning as Planning}
% \title{\sys: Automatic Data Cleaning as Optimization}
%\title{\sys: A Declarative Data Cleaning System Inspired By AlphaGo}
\title{AlphaClean: A Planning Approach Towards Unified Automatic Data Cleaning}

\numberofauthors{1}
\author{ Sanjay Krishnan$\,^{*}$, Eugene Wu{$\,^{\dag\dag}$, Michael J. Franklin$\,^{*\dag}$, Ken Goldberg$\,^{*}$}  \\
\affaddr{ $^*$UC Berkeley, ~~ $^\dag$University of Chicago, ~~ $^{\dag\dag}$Columbia University} \\
\affaddr{ \{sanjaykrishnan, franklin, goldberg\}@berkeley.edu ~~ ewu@cs.columbia.edu}\\
\affaddr{}
}

%\fontsize{9pt}{11pt}
%\selectfont


\maketitle

\begin{abstract}
Results in AI, like AlphaGo, have shown that a combination of Machine Learning and distributed search can effectively optimize very complex objective functions, e.g., the value of board positions in a complex game like Go.
These results are highly relevant to the study of data cleaning, which often involves an optimization over modifications to a dataset to enforce a complex, interdependent set of data constraints.
This paper explores explicitly posing the data cleaning as a planning problem: given an objective function that defines cleanliness, find a feasible sequence of data transformations that maximizes this objective.
Our system, \sys, is based on this general optimization framework that can express a very large class of data cleaning problems including cleaning with rules, statistical models, and crowd sourcing.
While very general, special case performance can be improved through an API that specifies pruning and data partitioning rules on top of this general framework. 
The system additionally adaptively learns additional pruning functions with Machine Learning to exploit systematic errors in the dataset--and re-use previously discovered transformation patterns.
The benefits of this approach are: (1) there are few restrictions on what the user can express at the cost of run time, (2) the output is a sequence of transformations rather than a clean database instance so the logic of the algorithm is more interpretable and can apply to future data, and (3) the optimization algorithm is greatly simplified and parallelizable.
Surprisingly, we find that \sys achieves parity with state-of-the-art specialized constraint, statistical, and quantitative data cleaning systems in terms of accuracy.
On two datasets considered in prior data cleaning work of Flight arrival times and a Physician registry, \sys achieves a similar precision and recall to a recently proposed Denial Constraint systems. 
Similarly, we applied \sys to numerical outlier problems and compared to the Minimum Covariance Determinant (MCD) algorithm.
Outliers removed by \sys  improved the accuracy of a downstream predictive models more significantly than MCD (5\% prediction accuracy improvement on the US Census Dataset) and a (4\% on an EEG dataset). 
We find that \sys achieves these levels for a very reasonable amount of additional computation (up-to 64 additional search threads).
\end{abstract}


%\pagenumbering{gobble}


\section{Introduction}\label{intro}\sloppy

It is widely known that data cleaning is one of the most time-consuming steps of the data analysis process~\cite{nytimes}, and designing algorithms and systems to automate or partially automate data cleaning continues to be an active area of research~\cite{DBLP:conf/sigmod/ChuIKW16}.  Automation in data cleaning is challenging because real-world data is highly variable.  A single data set can have many different types of data corruption such as statistical outliers, constraint violations, and duplicates.  Once an error is detected, there is a further question of how to repair this error, which often depends on how the data will be used in the future.

\ewu{Make clear why transformation and data quality measure is different}

Data cleaning is particularly challenging for data scientists and modern data analysis due to several trends.  First, data is increasingly collected across many different data sources; each dataset can exhibit different types of errors that are domain specific.  These error types can be {\it syntactic}, due to e.g., data extraction bugs that cause multiple attributes to be concatenated into a single string, or schema errors that cause entire columns to be shifted.  They can also be {\it semantic}, such as incorrect values that must be imputed, functional dependency violations, or entity resolution violations.   Each type of error, in each domain, often requires different methods to fix them.  These can range from complex imputation algorithms that use domain-specific error models (ref signal processing and holoclean), to simple rules that are easy to report and understand~\cite{}.  \ewu{For each, there are numerous specialized solutions}

Second, datasets often contain mixtures of the above errors.  These errors may be introduced at any part of the data extraction, cleaning, and processing pipeline before the devolper analyzes it.  For instance, data entry errors or incorrect survey coding may introduce semantically incorrect values for some attribute values.  In addition, data extraction or schema alignment algorithms may introduce syntactic errors where attributes are shifted or their values are concatenated.  Further, these errors are not independent---improper extraction affects the ability to interpret semantic values.  

Third, developers and users often do not know the quality characteristics and constraints of their data up front.  Instead, they often start with vague notions of what signals are associated with high quality data, and refine their understanding throughout the data cleaning process~\cite{}.  This makes it difficult to use super specifalized data cleaning applications, because it is easy to overinvest time into a a single tool. \ewu{Better argument?}  \ewu{Glue story to make the systems working together}

\ewu{human interfaces with quality metrics, mechanisms are data transformations, alpha clean works to combine these together.}

For this reason, it is desirable to have a system that optimizes the human-in-the-loop process.  The inputs to the system should let users specify, and incrementally refine, a high-level {\it data-quality measure} as well as the transformation operations that are applicable for her application domain.  \reminder{Examples of data quality measures and transformation operations.  Make sure xforms are parameterized}

The outputs of the system should be easy to understand.  Existing work suggests that simple programs composed of user-understandable operators are more interpretable than directly updating the database with the resulting fixes.  

As a concrete example, consider the following example of addresses in Figure~\ref{f:example-data}.  \reminder{SANJAY DESCRIBE ITS DATA QUALITY ISSUES.  Emphasize that the quality issues are complex, learned during the cleaning process, and cannot be captured by simple FD-style constraints.  }

In this example, no single data cleaning system is capable of automatically cleaning the errors.  Instead, the user must cobble together a variety of specialized data cleaning tools in the appropriate sequence.

\stitle{A Simple, General System}
In this paper, we take a step back from the specialized systems approach, and ask if a simple, general meta-algorithm can be used to address common data cleaning problems.  Our primary hypothesis is that data cleaning errors are typically highly structured, and a learning-based approach that can identify these structures for different cleaning problems.   
Data errors are often systematic where they are correlated with with certain attributes and values in the dataset~\cite{rekatsinas2017holoclean,DBLP:journals/pvldb/KrishnanWWFG16}.
Consequently, as more data is cleaned, we can better identify common patterns to prioritize the search on future data.

To this end, we design and evaluate \sys, which employs a simple tree search algorithm that finds a sequence of data transformation operations to maximize a user-specified data quality measure.   The user provides parameterized operators and the system identifiez the specific parameterizations throughout the search process.  In our experiments, we show how external data cleaning libraries such as \ewu{XXX and YYY} can be wrapped as parameterized black-boxes.  Similarly, the user can incrementally add additional constraints and conditions to the data quality measure.  These can be expressed as generic Python functions over a dataset, and we describe optimizations based on the distributive or algebraic properties of commen quality measures that capture e.g, entity resolution, functional dependencies.

As \sys searches possible programs by executing candidate operations and computing the resulting quality measure, it learns a prediction model to estimate the expected improvement of different transformation operators to quickly prune the search space.  In our work, we show that a simple linear classifier is effective for a wide variety of data cleaning datasets used in the current literature, however more complex prediction models such as neural networks~\cite{} may be used for more complex or widely used problems that can generate lots of training data.   A benefit of learning these models is that \sys can bootstrap new data cleaning problems by using previously learned models; we show how \sys can more quickly generate data cleaning programs when users can iteratively refine their data quality measures.

Finally, a benefit of a simple search algorithm is that it is flexible in the optimization criteria, the library of transformation operations that are supported, and is highly parallelizable.  The latter aspect has the capability of leveraging modern cloud and GPU infrastructure to cheaply parallelize the search.  Our experiments show near-linear scalability as the number of nodes increases.
Across 8 real-world datasets used in prior data cleaning literature, we show that \sys matches or exceeds the cleaning accuracy and exhibits competitive run-times to state-of-the-art approaches that are specialized to a specific error domain (constraint, statistical, or quantitative errors).  

The highlight of \sys is its flexibility to the variety and combination of data error types and cleaning transformation.   To show that the general approach does not degrade data cleaning quality too much, we compared with recently reported results from state-of-the-art systems, \sys performs comparably in accuracy and performance.


\noindent To summarize, our contributions include:
\begin{itemize}[leftmargin=*, topsep=0mm, itemsep=0mm]
  \item The design and evaluation of a general search-based approach to data cleaning, that combines automated cleaning for arbitrary combinations of syntactic and semantic data errors.
  \item The development of decomposable data cleaning measures that are amenable to parallel execution.
  \item A suite of pragmatic optimizations, such as fast pruning using predictive models, multi-node parallelization, and data sharing to reduce network communication bottlenecks, that reduce the runtime by \ewu{XXX$\times$}.
  \item A systematic study of the benefits and limitations of \sys in terms of data cleaning accuracy (precision, recall), and the runtime.  We show that \sys can solve incremental refinements of the data quality measure $yyy\times$ faster than from scratch, and that \ewu{SOME OTHER FINDING}
  % \item Finally, we show that \sys can solve existing data cleaning benchmarks at competitive runtimes and accuracies as existing specialized data cleaning systems.  We believe this is not necessarily due to superiority of \sys, but may be a symptom of limitations of existing cleaning benchmarks, which have simple structure.

\end{itemize}



\section{Problem Definition}
First, we overview the basic formalism of \sys and present its relationship to related work.

\subsection{Data Transformations}
We focus on data transformations that concern a single relational table. 
Let $R$ be an instance over a set of attributes $A$, and let $\mathcal{R}$ denote the set of all possible instances over $A$.
Let $r.a$ be the attribute value of $a \in A$ for row $r \in R$.
$T(R): \mathcal{R} \mapsto \mathcal{R}$ is a data transformation that maps an input relation instance $R \in \mathcal{R}$ to a new (possibly cleaner) instance $R' \in \mathcal{R}$ that adheres to the same schema.  For instance, ``replace all \texttt{city} attribute values equal to {\it San Francis} with {\it SF}'' may be one data transformation, while ``delete the $10^{th}$ record'' may be another.   Data transformations can be composed using the binary operator $\circ$:
\[
(T_i \circ T_j)(R) =  T_i(T_j(R))
\]
The composition of one or more data transformations is called a {\it cleaning program}.   

In practice, users will specify {\it transformation templates} $T(R, [\theta_1,\cdots,\theta_k])$, and every assignment to the parameters represents one possible transformation.  
\begin{example}\label{ex1}
The following relation contains two attributes \textsf{city\_name} and \textsf{city\_code}.  Suppose there is a one-to-one relationship between the two attributes. In this case, the relation is inconsistent with respect to the relationship and contains errors highlighted in \red{red}.

  \begin{table}[ht!]
  \centering
  \label{my-label}
  \begin{tabular}{|l|l|l|}
  \hline
  \rowcolor[HTML]{000000} 
  & \white{city\_name}            & \white{city\_code}   \\ \hline
  1 & San Francisco                    & SF                                  \\ \hline
  2& \red{\textbf{New York}}           & NY                                  \\ \hline
  3 & New York City                    & \red{\textbf{NYC}} \\ \hline
  4 & \red{\textbf{San Francisc}}      & SF                                  \\ \hline
  5 & San Jose                         & SJ                                  \\ \hline
  6 & San Mateo                        & SM                                  \\ \hline
  7 & New York City                    & NY                                  \\ \hline
  \end{tabular}
  \end{table}

The following transformation template uses three parameters: \texttt{attr} specifies an attribute, \texttt{srcstr} specifies a source string, and \texttt{targetstr} specifies a target string.   
\[
\textsf{find\_replace}(\text{srcstr}, \text{targetstr}, \text{attr})
\]
Given an input relation, the template finds all \texttt{attr} values equal to \texttt{srcstr} and replaces those cells with \texttt{targetstr}. 
For instance, \texttt{find\_replace(``NYC'', ``NY'', ``city\_code'')} defines a data transformation that fixes the error in the second attribute.
\end{example}

Let $\Sigma$ be a set of distinct data transformations $\{T_1,\cdots,T_N\}$, and
$\Sigma^*$ be the set of all finite compositions of $\Sigma$, i.e., $T_i\circ T_j$.
A formal language $L$ over $\Sigma$ is a subset of $\Sigma^*$.

\begin{example}\label{ex2}
  Continuing \Cref{ex1}, $\Sigma$ is defined as all possible parameterizations of \texttt{find\_replace}.  Since many possible possible parameterizations are non-sensical (e.g., the source string does not exist in the relation), we may bound $\Sigma$ to only source and target strings present in each attribute's instance domain~\cite{workthatdoesthis}.  In this case, there are $61$ possible data transformations, and $\Sigma^*$ defines any finite composition of these $61$ transformations.  The language $L$ can be further restricted to compositions of up to $k$ data transformations.  
\end{example}

Finally, let $Q(R): \mathcal{R} \mapsto [0,1]$ be a quality function where $1$ implies that the instance $R$ is clean.
Although this general formulation captures arbitatry forms of evaluation, including crowdsourced quality validation~\cite{tamr,stuff}, common data cleaning algorithms~\cite{} evaluate the cleanliness of in instance by evaluating the relation on a row-by-row or cell-by-cell basis.  For this reason, we define two classes of quality functions: row-separable and cell-separable quality functions.
The former expresses the overall quality based on row-wise quality function $q(r): R \mapsto [0,1]$ where $1$ implies that the record is clean:
\[Q(R) \propto \sum_{r \in R} q(r)\]
\noindent Similarly, a cell-separable quality function can be expressed based on a cell-wise quality function $q(r, a): (R\times A) \mapsto [0,1]$:
\[Q(R) \propto \sum_{r \in R} \sum_{a \in A} q(r,a)\]
\noindent We are now ready to present data cleaning as the following optimization problem:
\begin{problem}[Cleaning Synthesis Problem]
Given a quality function $Q$, a relation $R$, and a language $L$, find a sequence of transformations $l \in L$ (a sequence of inputs to the DFA) that optimizes the quality function.
\[
\textsf{opt}(Q,R,L) = ~ \min_{l \in L} Q( l(R) ).  
\]
\end{problem}

\begin{example}\label{ex3}
Continuing~\Cref{ex1}, let us assume the following functional dependencies over the example relation: $\textsf{city\_name} \rightarrow \textsf{city\_code}$ and $\textsf{city\_code} \rightarrow \textsf{city\_name}$.
We can efficiently identify inconsistencies by finding the cities that map to $>1$ city code, and vice versa.   Let such city names and codes be denoted $D_{city\_name}$ and $D_{city\_code}$, respectively.
$Q(R$ is a cell-separable quality function where the cell-wise quality function is defined as $q(r, a) = 1 - (r.a \in D_a)$, such that $r.a$ is $1$ if the attribute value does not violate a functional dependency, and $0$ otherwise.

By searching through all possible programs up to length \ewu{XXX} in $L$, we can find a cleaning program based on \texttt{find\_replace} that resolves all inconsistencies:
\begin{lstlisting}
    find_replace(New York, New York City, city_name)
    find_replace(San Francisc, San Francisco, city_name)
    find_replace(NYC, NY, city_code)
\end{lstlisting}
\end{example}


\subsection{Approach Overview and Key Challenges}
Talk about planning, incremental search over $L$, hardness.

\vspace{0.5em} \noindent \textbf{Remark 2: } \textsf{opt(Q,R,L)} is a hard problem. In fact, we can show that it is APX-Hard--that is, unless P=NP there does not exist a polynomial time approximation scheme.
Let $R$ be a single-attribute relation of Booleans. Let $L$ be the set of all assignments to a single value.
Given a list of $N$ Boolean clauses over all the boolean variables, let $Q$ assign to each record one minus the fraction of clauses that evaluate to true. This formulation is equivalent to MAX-SAT and solution to the optimization problem.




\if{0}
Every data cleaning problem in \sys is specified by a deterministic finite automaton (DFA). 
A DFA is a 5-tuple:
\[\langle S, A, \delta, s_0\rangle,\]
where $S$ is a set of states that the process can be in, $A$ is a set of inputs that the process can take, $\delta$ is a transition function that takes as input a state and an input and transitions the process to the next state in $S$, and $s_0$ is an initial state of the process.

The set of relations and the language of transformations defines a DFA:
\[\langle \mathcal{R}, \Sigma, \delta, R_{dirty}\rangle, \]
where the states $\mathcal{R}$ is the set of possible instances, the inputs are transformations from $\Sigma$, the transition function updates the instance with the transformation, and the initial state is the dirty instance $R_{dirty}$. Search problems can be defined over the DFA. 
A quality function $Q$ maps an instance $R$ to a scalar where 1 implies it is clean:
\[
Q: \mathcal{R} \mapsto [0,1]
\]
A separable quality function is one that can be expressed as an average over cell-wise quality metrics $q(r,a)$ where 1 implies clean:
\[
Q(R) \propto \sum_{r \in R} \sum_{a \in A} q(r,a)
\]

Note that in contrast to traditional programming-by-example problems~\cite{}, which take as input a well defined input and goal state, the latter is not present in our problem formulation.  It is replaced with a quality function that we seek to maximize.  This is an important distinction because (1) it models the uncertain nature of data cleaning, where the ground truth is typically not available~\cite{}, and (2) necessarily greedy.

\vspace{0.5em} \noindent \textbf{Remark 1: } For readers familiar with stochastic processes, this DFA is equivalent to a deterministic Markov Decision Process (MDP), where the states are $\mathcal{R}$, the actions $\Sigma$, the transition function updates the instance with the transformation, the initial state is the dirty instance $R_{dirty}$, and the reward function is $Q$. MDPs are usually stochastic and as such the optimal solution in general is not a sequence but a function from states to actions.



\fi

\section{Taxonomy of Related Systems}
Since the beginning of data management, several research and commercial systems have been proposed to improve data cleaning efficiency and accuracy (see~\cite{rahm2000data} for a survey).
Next, we describe how different data cleaning paradigms can be cast as a sequential search problem.

\subsection{Constraint-based Systems}
The classical model for data cleaning is using integrity constraints. 
The data scientist describes the relationships between attributes in terms of functional dependencies, conditional functional dependencies, denial constraints, and other types of logical constraint constraints.
Then, if any of these constraints are violated, the system will search over updates to the inconsistent database instance to enforce the constraints.
Recent example systems include NADEEF~\cite{DBLP:conf/sigmod/DallachiesaEEEIOT13}, LLunatic~\cite{geerts2013llunatic}, Holistic Data Cleaning~\cite{chu2013holistic}, and Big Dansing~\cite{khayyat2015bigdansing}.
As in the running example, we can construct a quality function that marks every cell that is in violation with a 0 and every cell that is not with a 1.
The allowed language would be replacing any cell with another value in the attribute domain.

While prior work are roughly speaking equivalent in expressiveness, they each have a different solution algorithm.
For example, Dallachiesa et al. uses a SAT solver to enforce the constraints~\cite{DBLP:conf/sigmod/DallachiesaEEEIOT13}, Chu et al. uses a iterative algorithm that walks along a hypergraph~\cite{chu2013holistic}, and Geerts et al. is based on a fixed-point iteration~\cite{geerts2013llunatic}.
Our goal with \sys is to reduce the complexity of such systems, by consolidating to a single general-purpose search algorithm and an API for specifying optimizations specific to particular problem classes.
This means that the approach is more general and can handle novel combinations of data cleaning operations, e.g., denial constraints and numerical outliers.

\subsection{Model-based Systems}
Another class of systems uses statistical properties to define data errors, which is also called Quantitative Data Cleaning (see survey by Hellerstein~\cite{hellerstein2008quantitative}).
A user can define a statistical model that the data should conform to, e.g., all numerical values must be concentrated with $p$ standard deviations from the mean. 
More generally, one can use a model to score records to see how likely those value combinations are.
When presented with data that explained well by this model, the system tries to replace the value.
Example systems include Eracer~\cite{eracer}, SCARED~\cite{yakout2013don}, and dBOOST~\cite{pit2016outlier}.

Such scoring functions naturally fit into the model of \sys.
A statistical model can define the quality function and the user can chose preferred behavior for abnormal records.
For example, one can chose to delete or impute with a default value. Each cell is scored by a an anomaly detector which marks a percentage confidence that the element is an anomaly (e.g., by the distance to the nearest neighbor). The quality function is the average score over the entire table.
The search problem is to find a sequence of transformations that reduces the anomaly detection scores.

Similar techniques can be applied to non-numerical data if an appropriate featurization is used.
One approach is to borrow recent results from Natural Language Processing using Neural Networks to first embed the records in a vector-space and then apply numerical outlier detection techniques. The \textsf{word2vec} model \cite{mikolov2013distributed} is one such approach.
Using large amounts of unannotated plain text, \textsf{word2vec} learns relationships between words automatically with a Neural Network that predicts the occurrence of nearby words.
 Each word is assigned a vector in the vector space such that words that share common contexts (i.e., occur in the same document) in the corpus are located in close proximity to one another in the space.
 This vector space captures semantic relationships between words.
 
Each record is treated as a document and each attribute is treated as word.
 The model is then trained using all of the records in the training dataset.
 Thus, for each attribute value we have a vector.
 To featurize a record, we concatenate these vectors together.
 Therefore, for each record $r$ there is an associated vector $r_v$.
  Like the NLP application, this vector space captures semantic relationships between records.
  We can define the same anomaly detection objective as above over these features.
  
\subsection{Programming-by-Demonstration}
Finally, \sys is very related to systems that apply programming-by-demonstration (PbD) approach to data preparation problems.
This approach was most notably proposed in the Data Wrangler project, where a human provided initial examples of how to transform a set of semi-structured tuples into a structured schema~\cite{wrangler,trifacta}.
The system performed a search to find a sequence of transformations to best reproduce the humans demonstration.
Recently, this basic approach has been extended in the Foofah system~\cite{jin2017foofah}.

Such approaches are clearly linked to \sys. The quality function is a score on each record on how accurately it matches a manually cleaned gold-standard. \sys can additionally couple automated and PbD approaches. 
Consider the example table as before (where managers can't earn less than employees), but this time  a human provides an example repair.
The quality function measures the degree to which the table matches the human examples after applying the data transformations.









\begin{figure}[t]
\centering
 \includegraphics[width=0.7\columnwidth]{figures/arch}
 \caption{\small \sys decouples sampling from the parameter space from search. This allows the user to iterate quickly by observing early best-effort results. \label{fig:arch}}
\end{figure}



\section{Approach Outline}

\sys takes as input a user-provided quality function $Q$, and searches through the space of candidate pipelines $\mathcal{P}$ to find a cleaning pipeline $p*\in\mathcal{P}$ that maximizes the resulting quality of the relation $Q(p*(R))$.  \sys is progressive: at any time, \sys reports the best cleaning pipeline found so far.  This helps the user trade-off between search time and cleaning quality.  This section provides an architectural overview, and the subsequent section dive deeper into the implementation of the search algorithm and learning-based search-space pruning.

% Each conditional assignment can be appended to the current pool of candidate pipelines,  and then extend the current pool of candidate pipelines with the new condi  Thus the cost to execute a given cleaning operator can be an enormous bottleneck on the ability to generate candidate pipelines.  

%The space of possible conditional assignments and cleaning pipelines is far too large, and only a small subset is feasible to materialize at any time. 



\subsection{Synchronous Tuning}

Recall that a pipeline is composed of conditional assignments, which are generated by executing cleaning operators with specific parameter values.   A naive approach would select an operator $o$, draw a sample $\phi$ from its parameter space, execute $o(\phi)$ to generate a set of conditional assignments.      It then composes each conditional assignment $ca_i$ to each pipeline $p_j$ in the current pool of candidates, executes the new pipeline $p' = ca_i \circ p$, and evaluates the quality function $Q(p'(R))$.

Blackbox search algorithms typically couple the candidate (condition assignment) generation step with the quality evaluation step, because the former is usually very fast whereas quality evaluation may be slow.  In our context, the opposite is true, and quality evaluation is much faster than candidate generation.   Thus, the naive approach is susceptible to blocking on straggler cleaning operators that take a lon time to generate conditional assignments. 

\subsection{Asynchronous Architecture}

We propose a generate-then-search framework, that decouples the execution of cleaning operators and pipeline quality evaluation (Figure~\ref{fig:arch}).  The {\it Searcher} uses the {\it Parameter Sampler} to sample from each operator's parameter space.  {\it Library} takes these parameters and executes each operator $O_i$ in a separate process, which adds to a shared pool of conditional assignments.  The Searcher draws from this pool to create cleaning pipelines, and sends the best pipelines so far to the user.  The details of the search implementation is described in Section~\ref{s:search}.   The {\it Quality Evaluator} uses incremental view maintenance to efficiently evaluate the quality function for new pipeines based on the records that have been modified, and is described below.  Finally, the {\it ML-based Pruner} prunes the parameter search space using a machine learning model and  Section~\ref{s:pruning} describes it in detail.

One benefit of this asynchronous approach is that the search process does not block on a straggler cleaning operator.  It is common that operator parameters affect their runtime.  For example, inference thresholds and partitioning parameters can have ``cliffs'', where a small change in parameters can drastically slow down the performance of the method. Including such parameter settings in the search process naively would block the entire system.  In contrast, \sys will simply sample from faster operators until the slow inference task completes.  

One drawback of the asychronous approach is that the number of quality function evaluations is much higher\ewu{WHY?}. We argue that this is acceptable in many data cleaning settings, since quality function evaluation is often far cheaper than candidate repair generation (e.g., checking integrity constraints is cheaper than enforcing them). Furthermore, 


Our implementation simply allocates one process per operator, however this approach explicitly highlights the connection between the search space that is explored and resource scheduling.  For instance, allocating more CPU resources to slower cleaning operators would lead to a more uniform exploration of the search space.  We defer such investigation to future work.



% Our main architectural insight is a generate-then-search framework.  Possible parameter values are fed into a library of data cleaning methods, and these parameter assignments asynchronously generate candidate repairs to the dataset (called conditional assignments).  In parallel, a search thread builds a data cleaning plan using compositions of those transformations in the set.  Existing search algorithms implicitly pipeline these two steps; whereas, we decouple these two steps where candidate repairs can be generated in a separate thread and the search algorithm can proceed independently.  This contrasts from the baseline architecture, hereafter called \emph{synchronous tuning}, where a hyperparameter tuning algorithm will then select and assign parameter values to a sequence of operators and evaluate the quality at the end.  There are a few benefits for the asynchronous architecture of \sys.


% \vspace{0.5em} \noindent \textbf{Benefit 2. Progressive Results: } Next, the decoupling also allows for improved progressive behavior with early results. The search thread continuously polls the candidate conditional assignment set for new expansions. The naturally faster  data cleaning methods (e.g., approximate) will generate candidate repairs faster. Slower methods will eventually add these candidate repairs. This allows users to identify faults or glitches in their quality specifications or parameter spaces more quickly than if all methods were synchronously tuned in a pipeline.

\subsubsection{Parameter Space}
By default, users simply specify a list of allowable values for each operator parameter, and \sys samples from their values.  In addition, users can specify two types of parameters, for which \sys can apply search optimizations:

\begin{itemize}[leftmargin=*,topsep=.3em,itemsep=-.2em,partopsep=-.5em]
  \item \stitle{Attribute Name Parameters} If the parameter represents an attribute in the database, then \sys can infer the domain of allowable values.   For example, a numerical outlier detection algorithm might apply to a single attribute or a subset of attributes. \sys can also prune the paramater space by pruning attribute names that are irrelevant to the quality function.  

  \item \stitle{Threshold Parameters} Numeric parameters are often used as thresholds, inference parameters, or confidence bounds.  For these, users specify the most and least restrictive ends of the value domain, and \sys will sweep the space from most to least restrictive.   For instance, \texttt{ispell} only uses the dictionary if the attribute value is within \texttt{rec} characters of the dictionary word.  Thus, \sys will initially sample $\texttt{rec}=0$ and gradually relax the threshold.   
\end{itemize}

%which have the property that if the quality function    Another broad class of parameters in data cleaning methods are numerical parameters like thresholds and inference parameters. For these parameters, the user specifies a range (or a multi-dimensional grid). Often times, these parameters correspond to confidence metrics. In the example with the \texttt{City} table, the \texttt{ispell} method has a acceptance threshold for spell-checking. In these cases, we recommend that this search space is ordered fine-to-coarse. Where the most restrictive threshold is evaluated first towards increasingly less restrictive thresholds. 


\subsubsection{Incremental Quality Evaluation}

Most cleaning operators modify significantly fewer records than the entire dataset.  Since quality functions are simply aggregation queries, \sys can incrementally evaluate the quality function over the fixed records rather than the full dataset.  
This is exactly the process of incremental view maintenance, and we use standard techniques to incrementally compute quality functions that \ewu{HAVE X and Y properties}.  

Suppose we have relation $R$, quality function $q$, and a set of conditional assignment expressions $C$.  When possible, \sys computes $q(R)$ once and then for each of the expressions $c \in C$ compute a delta such $q(c(R)) = q(R) + \delta_c(q(R))$.
For many types of quality functions such incremental computation can be automatically synthesized and can greatly save on computation time.

Let us consider a concrete example with the quality function $q_1$, a functional dependency checker, from the previous section.
$R'$ is the resulting relation after applying \texttt{c} to all of the records.
Let $r_{pred}$ be the set of records that satisfy the predicate of the conditional assignment expression and $r_{pred}'$ be the resulting transformed records.
$q_1$ can be expressed in relational algebra in the following way:
\[
q_1(R') = \textsf{count}( R' \bowtie R' )
\]
$R'$ can be described in terms of $R$:
\[
R' = R - r_{pred} + r_{pred}' 
\]
leading to the following expression:
\[
q_1(R') = \red{q_1(R)} - \textsf{count}( r_{pred} \bowtie R )  + \textsf{count}( r'_{pred} \bowtie R )
\]
If we used a hash join to evaluate this quality function, the cost of incrementally maintaining is roughly linear in the size of the number records changed rather than the size of the relation.

% The consequence is that evaluating changes in quality can be very efficient for a large class of data cleaning problems.  The architecture of \sys is designed to exploit this property.  We assume conditional assignment generation to be expensive but quality evaluation to be fast.  An architecture that runs the entire pipeline before seeing the resultant quality is wasteful.




\section{Search Algorithm}
The search algorithm takes a quality function, a language, and a dirty relation, and outputs a sequence of transformations (of max depth $k$) that maximizes the quality function.

\begin{algorithm}[t]
\KwData{Q, R, L, (k, $\gamma$)}

Initialize $O$ as a priority queue with a singleton NOOP transformation\\

\While{ $\exists ~ o \in O: \|o\| < k$ }
{
    \For{$o \in O: \|o\| < k$}{
        
        Pop $o$ from the queue.
        
        \For{$t \in L$}{
             $o' = o \circ t$
             
             Push $o'$ with priority $\bar{Q}(o'(R))$.
        }
    }
    
    Pop all elements from with a priority greater than $\gamma$ times the lowest value in the queue.
}

\Return Lowest item on the queue
\caption{Greedy Best-First Tree Search}
\label{alg:main}
\end{algorithm}

\subsection{Motivation}
In principle, any tree search algorithm can apply over the data cleaning DFA and would be correct.
However, the traversal order and expansion policy is important in this search problem and can lead to highly suboptimal performance.

Let us first start with a breadth-first search (BFS). At each round one would expand every node on the frontier by composing it with another transformation in the library.
The first problem with this algorithm is that since each node in this tree $o$ represents a sequence of transformations.
Evaluating the value of $o$ can be very expensive since it would have to evaluate the entire path to the root.
$o$ is a composition of many transformations and may require a number of passes over the dataset.
This can be avoided if we can materialize (either to disk or memory) the frontier,that is, for each node in the priority queue $o \in O$, we have a cached result of $o(R)$. 
However, with BFS, the frontier is expoential in the support of the language and the system would quickly run out of memory.

An algorithm with slightly better system properties would be a depth-first search (DFS).
Cache would only have to maintain the preceding $k-1$ nodes on the path to the root.
However, DFS will waste time on highly improbable sequences of transformations.

\subsection{Basic Algorithm}
A best-first search expands the most promising nodes chosen according to a specified cost function.
We consider a greedy version of this algorithm, which removes nodes on the frontier that are more than $\gamma$ times worse than the current best solution.
Making $\gamma$ larger makes the algorithm asympotically consistent, whereas $\gamma=1$ is a pure greedy search.

The algorithm is described in Algorithm \ref{alg:main}.
The algorithm is initialized with an identity transformation. This identity transformation is placed on a priority queue where the priority is the aggregate quality after applying the transformation (in this case the quality of the original relation $R$).
Then, the algorithm ``expands'' all elements on the queue with description length of less than $k$.
By expansion, we mean that it removes the element from the queue composes the element with a transformation from the library.
Then, it places the new composed transformation onto the priority queue with its new quality score as the priority.
The algorithm then flushes the priority queue of all elements with priority more than $\gamma$ times worse than the current best solution.
This process repeats until all elements on the queue have a description length of $k$.

We can materialize (either to disk or memory) the frontier,that is, for each node in the priority queue $o \in O$, we have a cached result of $o(R)$. 
Then, when we expand the nodes to $o' = o \circ t$, we only have to incrementally evaluate $t(R)$.
After the node is expanded, the result is added to the cache if it within $\gamma$ of the best solution.
The basic algorithm described above is well-suited for this problem.
Without the greediness, the frontier might be exponentially large leading to an impractical amount of materialization.
By tuning $\gamma$, the user can essentially set how much memory is used for materialization.

\begin{figure*}
    \centering
    \includegraphics[width=0.6\textwidth]{figures/distributed.png}
    \caption{TODO}
    \label{fig:my_label}
\end{figure*}

\subsection{Parallelism}
The best-first search algorithm also is amenable to parallelization. One can parallelize over the two inner for loops $O \times L$. Each expansion can be forked into its own thread. However, this actually makes the materialization described above a bit more challenging. 

\subsubsection{Shared Memory Parallelization}
The most straight-forward case is when we have access to low-latency shared memory between the expansion threads. In each expansion round, the main thread will assign each expansion node to workers and they will evaluate a given transformation. Each worker will make a copy of $o(R)$ (the node it is expanding) into shared memory. 
If the expanded transformation is within $\gamma$ of the best current result, then it will update the copy, otherwise delete it. The point of synchronization is after all expansions are finished, and the main driver will flush all transformations less than $\gamma$ of the best result, and then assign new workers for the next round. 

\subsubsection{Distributed Parallelization}
In cases, where we do not have access to low-latency shared memory, the communication costs of the above algorithm can be impractical. For each expanded node, the entire table has to be communicated to each of the distributed nodes.
We consider a worker-driver model, and assume that the number of workers is $k$.

\vspace{0.25em} \noindent  \textbf{Initialization. } 
Each worker has a copy of the base relation with no transformations.
In the first round of expansion, the driver assigns to each worker $\frac{|L|}{k}$ expansions. 
Each worker executes each of its assigned expansions, and stores the transformations that are within $\gamma$ of the best local result.

\vspace{0.25em} \noindent  \textbf{Reconciliation. }  Note that the global top-$\gamma$ set is a subset of the union of the all local results.
The workers then communicate the quality of their best transformation. This can be used to reconcile the local materializations to only the global top-$\gamma$ set.
This set is not necessarily balanced, e.g., one worker might have almost all of the top transformations.
The next step is a balancing step, where each worker communicates the number of materialized expansions it currently stores.
The workers with more than $\frac{|O|}{k}$ materialized expansions  randomly select ones to evict, and the driver re-distributes these to nodes with too few materialized expansions.
This is done by communicated the transformation and the result is re-computed on the new worker.
If $|O| < k$, then expansions are chosen at random to be replicated.
The result of the reconciliation step is that all workers have an evenly distributed set of materialized expansions.

\vspace{0.25em} \noindent  \textbf{Next Round. } After reconciliation, each worker is associated with a particular $o \in O$ (or a set of them). To parallelize, the driver must simply ensure that it assigns new expansions only to those workers that have materialized the parent.
The algorithm then repeats, expanding each node locally, and then reconciles the results.





\section{Search Optimizations}\label{s:opts}

\subsection{Parallelism}\label{s:parallel}
The best-first search algorithm also is amenable to parallelization. One can parallelize over the two inner for loops $O \times L$. Each expansion can be forked into its own thread. However, this actually makes the materialization described above a bit more challenging. We use Ray~\footnote{https://github.com/ray-project/ray} to implement the parallel search. 

\subsubsection{Shared Memory Parallelization}
The most straight-forward case is when we have access to low-latency shared memory between the expansion threads. In each expansion round, the main thread will assign each expansion node to workers and they will evaluate a given transformation. Each worker will make a copy of $o(R)$ (the node it is expanding) into shared memory. 
If the expanded transformation is within $\gamma$ of the best current result, then it will update the copy, otherwise delete it. The point of synchronization is after all expansions are finished, and the main driver will flush all transformations less than $\gamma$ of the best result, and then assign new workers for the next round. 

\subsubsection{Distributed Parallelization}
In cases, where we do not have access to low-latency shared memory, the communication costs of the above algorithm can be impractical. For each expanded node, the entire table has to be communicated to each of the distributed nodes.
We consider a worker-driver model, and assume that the number of workers is $k$.
A schematic diagram of the algorithm is shown in figure \ref{fig:algo}.

\vspace{0.25em} \noindent  \textbf{Initialization. } 
Each worker has a copy of the base relation with no transformations.
In the first round of expansion, the driver assigns to each worker $\frac{|L|}{k}$ expansions. 
Each worker executes each of its assigned expansions, and stores the transformations that are within $\gamma$ of the best local result.

\vspace{0.25em} \noindent  \textbf{Reconciliation. }  Note that the global top-$\gamma$ set is a subset of the union of the all local results.
The workers then communicate the quality of their best transformation. This can be used to reconcile the local materializations to only the global top-$\gamma$ set.
This set is not necessarily balanced, e.g., one worker might have almost all of the top transformations.
The next step is a balancing step, where each worker communicates the number of materialized expansions it currently stores.
The workers with more than $\frac{|O|}{k}$ materialized expansions  randomly select ones to evict, and the driver re-distributes these to nodes with too few materialized expansions.
This is done by communicated the transformation and the result is re-computed on the new worker.
If $|O| < k$, then expansions are chosen at random to be replicated.
The result of the reconciliation step is that all workers have an evenly distributed set of materialized expansions.

\vspace{0.25em} \noindent  \textbf{Next Round. } After reconciliation, each worker is associated with a particular $o \in O$ (or a set of them). To parallelize, the driver must simply ensure that it assigns new expansions only to those workers that have materialized the parent.
The algorithm then repeats, expanding each node locally, and then reconciles the results.









\subsection{Learning Dynamic Pruning Rules}\label{s:dynlearn}
To effectively search through the language of transformations, search heuristics are important.
However, it can be challenge to devise these heuristics \emph{a priori}.
This section describes how Machine Learning can be used to learn a search heuristic as data is cleaned.

\subsubsection{Motivation}
The search algorithms in  most automatic data cleaning frameworks are carefully tuned for a specific quality function or class of quality functions. For example, the chase used in functional dependency resolution does not make an edit to the table unless it enforces at least one tuple's FD relationship. Exploiting the structure of the specific problems allows for a tractable solution technique. Similarly, in entity matching problems, one restricts the search to only matching tuples that are likely to be similar based on some similarity metric.
These are not static optimizations, i.e., knowing how to prune the search space requires knowing the underlying data or making strong modeling assumptions about the types of transformations used.

\subsection{Algorithm \ewu{(Give a name?)}}
\sys executes the search on each block of data.
The result is a sequence of transformations to optimize the quality metric on that block.
Every transformation in this sequence can be treated as a positive training example $L^+$, and every transformation not in this sequence can be treated as a negative example $L^-$.
The idea is that if we apply the search to a sufficient number of blocks then we can train a classifier to predict whether a transformation will be included in the final sequence.
It is important to note that this prediction is over the transformations and not the data. 

Internally, \sys uses a Logistic Regression classifier to make this classification. The Logistic Regression is tuned towards False Positives (i.e., keep a bad search branch). This is done by training the model and shifting the prediction threshold until there are no False Negatives. 


\subsection{Requirements on Transformations}
Suppose we made only the following assumption about the language of transformations used.
Each transformation consists is described by a fixed-length feature vector in $\mathbb{R}^k$:
\[
\textsf{feat}: T \mapsto \mathbb{R}^k 
\]
An example of a featurization, consider a transformation from the the running example:
\begin{lstlisting}
find_replace(New York, New York City, city_name)
\end{lstlisting}
The featurization could encode the string similarity of the two literal parameters and an indicator vector describing which column it applies to.

Consider an alternative to a predefined search heuristic where we clean data in small blocks.
For the initial blocks, we search without a heuristic.
As we continuously perform the search, we train a classifier on these features to reject search branches that are not typically in the final solution.
This allows us to exploit any patterns in the literal parameters that repeatedly occur.

For example, some columns might not be dirty and are not worthwile to clean.
In the example above, another observation could be that the source and target strings in the optimal sequence are very close in terms of string similarity (as opposed to arbitrary transformations).
If each of these operations was featurized with a single scalar that is the edit-distance between the two strings, then the classifier could learn a pruning threshold (i.e., not considering find-and-replace operations above that threshold).



\section{Experiments}\label{s:exp}
We illustrate the performance of \sys on real and synthetic benchmark datasets.

\subsection{End-to-End Evaluation}

\subsubsection{Baselines}


\section{Related Work}
Since the beginning of data management, several research and commercial systems have been proposed to improve data cleaning efficiency and accuracy (see~\cite{rahm2000data} for a survey).
Over the past few years, there have been several significant data cleaning advances in scalability~\cite{wang1999sample, khayyat2015bigdansing, altowim2014progressive}.
However, \emph{machine time} is only part of the story and the \emph{human time} of data cleaning is known to be significant.
\sys aims to reduce the human time in writing data cleaning scripts by allowing the data scientist to model the data at a high-level and the system searches for the appropriate low-level transformations.

\vspace{0.25em} \noindent \textbf{Learning in data cleaning: } There are a number of other works that use machine learning to improve the efficiency and/or reliability of data cleaning~\cite{DBLP:journals/pvldb/YakoutENOI11,yakout2013don,gokhale2014corleone}.
For example, Yakout et al. train a model that evaluates the likelihood of a proposed replacement value \cite{yakout2013don}.
Another application of machine learning is value imputation, where a missing value is predicted based on those records without missing values.
Machine learning is also increasingly applied to make automated repairs more reliable with human validation \cite{DBLP:journals/pvldb/YakoutENOI11}.
Human input is often expensive and impractical to apply to entire large datasets.
Machine learning can extrapolate rules from a small set of examples cleaned by a human (or humans) to uncleaned data \cite{gokhale2014corleone, DBLP:journals/pvldb/YakoutENOI11}.
This approach can be coupled with active learning \cite{DBLP:journals/pvldb/MozafariSFJM14} to learn an accurate model with the fewest possible number of examples.
Holoclean~\cite{rekatsinas2017holoclean} leverages machine learning to validate repairs with a probabilistic graphical model.
\sys uses machine learning in the synthesis process to prune search branches.


\vspace{0.25em} \noindent \textbf{Generalized Definitions of Data Quality: } Also relevant to \sys, are works that consider generalized definitions of data quality--beyond integrity constraints.
There is a growing body of literature that studies analysis-driven data cleaning, that is, applying data cleaning in a sufficient way to answer a given query (defining quality flexibly as the accuracy of the query result).
For example, Altwaijry et al.~\cite{altwaijry2015query} describe a technique for resolving a sufficient subset of entities in a database to answer SPJ queries.
Bergman et al. \cite{DBLP:conf/sigmod/BergmanMNT15} proposed identifying errors in selection query results and generating crowd-scoured queries to determine fixes to the base data.
Similarly, work on the consistent query answering problem explored the minimal effort needed to answer a query given a set of integrity constraints over a dirty relation~\cite{DBLP:series/synthesis/2011Bertossi}.
While the work on relational queries is extensive, analytical queries (aggregates, advanced statistical analytics, learning etc.) is less studied.
Projects like ActiveClean~\cite{DBLP:journals/pvldb/KrishnanWWFG16} have studied algorithms for prioritizing user-defined cleaning using the downstream ML model.
\sys is a flexible framework where such cleaning objectives can be modeled easily as different quality functions.

\vspace{0.25em} \noindent \textbf{Languages in Cleaning: } Languages for data transformations are also a well-studied field. The seminal work on this subject was Raman and Hellerstein~\cite{raman2001potter} for schema transformations, and Galhardas et al.~\cite{DBLP:conf/vldb/GalhardasFSSS01}. These ideas were later extended in the Wisteria project~\cite{DBLP:journals/pvldb/HaasKWF015} to parametrize the transformations to allow for learning and crowdsourcing. Algorithmically, \cite{wrangler, jin2017foofah} are the closest relatives to \sys. These systems use human demonstrations of data transformations, and leverage a search over the language to mimic these demonstrations. This process is a special case of \sys.




\section{Conclusion and Future Work}
The research community has developed increasingly sophisticated data cleaning methods~\cite{dc, rekatsinas2017holoclean, DBLP:journals/pvldb/KrishnanWWFG16, DBLP:conf/sigmod/ChuIKW16, mudgal2018deep, doan2018toward}.
The burden on the analyst is gradually shifting away from the design of hand-written data cleaning scripts, to building and tuning complex pipelines of automated data cleaning libraries.
The main insight of this paper is that tuning pipelines of data cleaning operations is very different than tuning pipelines for machine learning.

Rather than treating each pipeline component as a black-box transformation of an entire table, \sys extracts the constituent replacements. Given a library of data cleaning methods, each suggests potential replacements policies which are aggregated into a central set. This defines a well-posed plan-space, namely, the set of all compositions of candidate functions. This plan space captures method reordering, method exclusion, and applying a method to a subset of records.
  
Although our results suggest that borrowing from recent advances in planning and optimization is a fruitful direction, the results are counter-intuitive and raise a number of questions about future opportunities in data cleaning.  Does \sys achieve its results benchmarks used in data cleaning are too simple and amenable to greedy brute-force search?  
We hope to really characterize and understand these tradeoffs in the future. We are excited to extend \sys towards a more flexible and usable data cleaning system.  In particular, data cleaning is inherently a visual and interactive process, and we plan to integrate \sys with a data visualization interface.   Users can visually manipulate and examine their dataset and the system can translate interactive manipulations into quality functions.  This will also require work to characterize failure modes and provide high level tools to debug such cases.  We are also hopeful that the compact codebase ($<$200LOC for the core search and learning algorithms) can enable more rapid development of specialized data cleaning systems for novel domains and error conditions.  



% The prevailing wisdom in the design of data cleaning algorithms is to exploit the details of specific problem rather than considering the most general cases, and our experiments suggest that this a general framework like \sys can achieve parity in terms of accuracy.
% While the serial implementation of \sys can be much slower than the competitor specialized frameworks, \sys can be efficiently distributed to significantly reduce the gap.
% 
% These results should be considered a proof-of-concept that such a data cleaning \sys can be built around the recent results in AI. 
% However, to us, these results are still counter-intuitive, and raise a number of speculative questions for the future: (1) are specialized systems overly engineered for worst-case guarantees and perhaps real-datasets are not that pathalogical, (2) maybe the benchmarks that we consider in data cleaning are too easy to brute-force, (3) what are the failure modes and corner cases of \sys in real data.
% We hope to consider these problems in future work, as well as extending the system to novel settings.
% In particular, we are interested in \sys as a middleware layer for data visualization.
% A user can manipulate data in a visual UI and these manipulations can be translated into a quality function.
% 

%\input{acks.tex}




%\bibliographystyle{abbrv}

{
\fontsize{8.8pt}{9.6pt} \selectfont
\small
\bibliographystyle{abbrv}
\bibliography{ref} 
\scriptsize
}


%
\vspace{0.5em}\noindent\textbf{USCensus: } This dataset contains US Census records for adults and the goal is to predict  whether the adult earns more than $50,000$ dollars. It contains 32,561 records with 15 numerical and categorical attributes. This dataset contained missing values and coding inconsistencies.
Examples of data error include:
\begin{lstlisting}
#missing values
40,Private,121772,Assoc-voc,11,
Married-civ-spouse,Craft-repair,Husband, 
Asian-Pac-Islander,Male,0,0,40,(*\orange{\bf{?}}*),>50K

#coding inconsistency
57,Local-gov,110417,HS-grad,9,
Married-civ-spouse,Craft-repair,Husband,
White,Male,(*\orange{\bf{99999}}*),0,40,United-States,>50K
\end{lstlisting}


\vspace{0.5em}\noindent\textbf{NFL: } This dataset contains play-by-play logs from US Football games. The dataset contains 46,129 records with 65 numerical, categorical, and string-valued attributes. Given the record, the classification objective is to determine whether the next play the team runs is a run or a pass play.
The dataset contains a significant number of missing values and ``sentinel'' records that mark the end of a log sequence. The sentinel records do not signify a play but rather signify a timeout, end of quarter, or end of the game.
\begin{lstlisting}
#missing values
"36",2015-09-10,"2015091000",1,1,(*\orange{\bf{NA}}*),"15:00",
15,3600,0,"NE",35,35,0,0,0,(*\orange{\bf{NA}}*),"PIT","NE"(*\blue{\bf{....}}*)

#sentinel record
"189710",2016-01-03,"2016010310",10,2,NA,"00:00",
0,1800,8,"GB",17,17,0,-1,0,0,"",NA,"END(*~*)QUARTER2"
,1,0,0,0,NA, NA,NA,0,"Quarter(*~*)End"(*\blue{\bf{....}}*)
\end{lstlisting}


\vspace{0.5em}\noindent\textbf{EEG: } This is a dataset of EEG recordings. 
The training data is organized into ten minute EEG clips labeled "Preictal" for pre-seizure data segments, or "Interictal" for non-seizure data segments. 
There are 2406 records each of which is a variable-length time-series of 16 attributes. We featurize this dataset into records of 32 attributes--the mean and variance over the length of the time-series. 
This dataset primarily contains numerical outliers, the clips have spurious readings.
\begin{lstlisting}
#Time t=46 Normal
[-41.53080368041992, -9.605541229248047, 
-55.74542999267578, 17.77084732055664,
-1.6866581439971924, 38.86453628540039, 
17.108707427978516, 26.545927047729492, 
-12.696817398071289, -12.703478813171387, 
56.78707504272461, 3.2556533813476562, 
22.688213348388672, -25.728403091430664, 
-10.142332077026367, -11.585281372070312]

#Time t=47 Abnormal
(*\orange{[0, 8, -10, 9, 18, 6, -8, -41, -26, -72, -19, 70, 129, 53, 31, -11]}*)
\end{lstlisting}

\vspace{0.5em}\noindent\textbf{Sensor: } The Intel sensor dataset contains 928,991 temperature, humidity, and light sensor readings a sensor deployment. The classification task is to predict whether the readings came from a particular sensor (sensor 49). This dataset primarily has numerical outliers.
\begin{lstlisting}
#Normal Record
49  -0.999750  12.862100  10.368300  10.438300  
11.669900 (*\orange{\bf{13.493100}}*)  13.342300  8.041690  
8.739010  26.225700  59.052800

#Spurious Record
49  1.175188  12.279100  8.849360  9.005830  
10.111700  (*\orange{\bf{378.750000}}*)  19.319400  15.916200  
37.631400  27.150100  53.403700
\end{lstlisting}

\vspace{0.5em}\noindent\textbf{Titanic: } This dataset contains 891 records from the Titanic manifest with 12 attributes. The classification objective is to determine whether the passenger survived or not. There are missing values and string formatting errors.

\begin{lstlisting}
#missing values
891,0,3,"Dooley, Mr. Patrick",male,
32,0,0,370376,7.75,(*\orange{\bf{--}}*),Q
\end{lstlisting}

\vspace{0.5em}\noindent\textbf{Housing: } The housing dataset contains 1460 records and 81 attributes of house price listings. The classification objective is to determine whether the listed house will be sold above 750000. 
This dataset contains missing values as well as numerical outliers.
\begin{lstlisting}
#missing values
(*\blue{\bf{....}}*)204,228,0,0,0,(*\orange{\bf{NA,NA}}*),Shed,350,11,2009,WD,
Normal,200000
\end{lstlisting}

\vspace{0.5em}\noindent\textbf{Retail: } The online retail dataset contains 541,909 records of online retail purchases with 8 attributes. The classification objective is to predict whether the purchase occurred in the United Kingdom.
This dataset contains numerical errors where some purchased quantities are reported as negative.

\begin{lstlisting}
#outliers
C536391,21980,PACK OF 12 RED RETROSPOT TISSUES
,(*\orange{\bf{-24}}*),12/1/10 10:24,0.29,17548,United Kingdom
\end{lstlisting}

\vspace{0.5em}\noindent\textbf{Federal Election Commission Contributions: } The FEC provides a dataset of election contributions of 6,410,678 records with 18 numerical, categorical and string valued attributes. This dataset has a number of errors. There are missing values, formatting issues (where records have the wrong number of fields causing misaligment in parsing), and numerical outliers (negative contributions).

\begin{lstlisting}
#missing values
C00458844,"P60006723","Rubio, Marco","RUCINSKI,
ROBERT","APO","AE","090960009","US ARMY",
"PHYSICIAN",100,08-MAR-16,(*\orange{\bf{``''}}*),(*\orange{\bf{``''}}*),(*\orange{\bf{``''}}*),"SA17A",
"1082559","SA17.1074981","P2016"

#rejected contributions double recorded
C00458844,"P60006723","Rubio, Marco","SWAID, 
SWAID N. DR.","BIRMINGHAM","AL","352660827",
"NEWOLOGICAL SURGERY ASSOCIATES","PHYSICIAN",
(*\orange{\bf{-400}}*),28-DEC-15, "REDESIGNATION TO GENERAL","X",
"REDESIGNATION TO GENERAL","SA17A",
"1047126","SA17.892835B","P2016"
\end{lstlisting}

\vspace{0.5em}\noindent\textbf{Restaurant Dataset: } The restaurant dataset has 758 distinct records and 4 attributes. This dataset has typically been used as a benchmark for entity resolution since records are duplicated with minor inconsistencies.
We designed a multi-class classification task to see if we could predict the city from record.
One of the major inconsistencies was additional attributes appended to the restaurant category.

\begin{lstlisting}
campanile,624 s. la brea ave.,los angeles,
american

grill  the,9560 dayton way,beverly hills,
american (*\orange{\bf{(traditional)}}*)
\end{lstlisting}


\vspace{0.5em}\noindent\textbf{Housing: } The housing dataset contains 1460 records and 81 attributes of house price listings. The classification objective is to determine whether the listed house will be sold above 750000. 
This dataset contains missing values as well as numerical outliers.

\begin{lstlisting}
#missing values
(*\blue{\bf{....}}*)204,228,0,0,0,(*\orange{\bf{NA,NA}}*),Shed,350,11,2009,WD,
Normal,200000
\end{lstlisting}

\end{document}
